
% Default to the notebook output style

    


% Inherit from the specified cell style.




    
\documentclass[11pt]{article}

    
    
    \usepackage[T1]{fontenc}
    % Nicer default font (+ math font) than Computer Modern for most use cases
    \usepackage{mathpazo}

    % Basic figure setup, for now with no caption control since it's done
    % automatically by Pandoc (which extracts ![](path) syntax from Markdown).
    \usepackage{graphicx}
    % We will generate all images so they have a width \maxwidth. This means
    % that they will get their normal width if they fit onto the page, but
    % are scaled down if they would overflow the margins.
    \makeatletter
    \def\maxwidth{\ifdim\Gin@nat@width>\linewidth\linewidth
    \else\Gin@nat@width\fi}
    \makeatother
    \let\Oldincludegraphics\includegraphics
    % Set max figure width to be 80% of text width, for now hardcoded.
    \renewcommand{\includegraphics}[1]{\Oldincludegraphics[width=.8\maxwidth]{#1}}
    % Ensure that by default, figures have no caption (until we provide a
    % proper Figure object with a Caption API and a way to capture that
    % in the conversion process - todo).
    \usepackage{caption}
    \DeclareCaptionLabelFormat{nolabel}{}
    \captionsetup{labelformat=nolabel}

    \usepackage{adjustbox} % Used to constrain images to a maximum size 
    \usepackage{xcolor} % Allow colors to be defined
    \usepackage{enumerate} % Needed for markdown enumerations to work
    \usepackage{geometry} % Used to adjust the document margins
    \usepackage{amsmath} % Equations
    \usepackage{amssymb} % Equations
    \usepackage{textcomp} % defines textquotesingle
    % Hack from http://tex.stackexchange.com/a/47451/13684:
    \AtBeginDocument{%
        \def\PYZsq{\textquotesingle}% Upright quotes in Pygmentized code
    }
    \usepackage{upquote} % Upright quotes for verbatim code
    \usepackage{eurosym} % defines \euro
    \usepackage[mathletters]{ucs} % Extended unicode (utf-8) support
    \usepackage[utf8x]{inputenc} % Allow utf-8 characters in the tex document
    \usepackage{fancyvrb} % verbatim replacement that allows latex
    \usepackage{grffile} % extends the file name processing of package graphics 
                         % to support a larger range 
    % The hyperref package gives us a pdf with properly built
    % internal navigation ('pdf bookmarks' for the table of contents,
    % internal cross-reference links, web links for URLs, etc.)
    \usepackage{hyperref}
    \usepackage{longtable} % longtable support required by pandoc >1.10
    \usepackage{booktabs}  % table support for pandoc > 1.12.2
    \usepackage[inline]{enumitem} % IRkernel/repr support (it uses the enumerate* environment)
    \usepackage[normalem]{ulem} % ulem is needed to support strikethroughs (\sout)
                                % normalem makes italics be italics, not underlines
    

    
    
    % Colors for the hyperref package
    \definecolor{urlcolor}{rgb}{0,.145,.698}
    \definecolor{linkcolor}{rgb}{.71,0.21,0.01}
    \definecolor{citecolor}{rgb}{.12,.54,.11}

    % ANSI colors
    \definecolor{ansi-black}{HTML}{3E424D}
    \definecolor{ansi-black-intense}{HTML}{282C36}
    \definecolor{ansi-red}{HTML}{E75C58}
    \definecolor{ansi-red-intense}{HTML}{B22B31}
    \definecolor{ansi-green}{HTML}{00A250}
    \definecolor{ansi-green-intense}{HTML}{007427}
    \definecolor{ansi-yellow}{HTML}{DDB62B}
    \definecolor{ansi-yellow-intense}{HTML}{B27D12}
    \definecolor{ansi-blue}{HTML}{208FFB}
    \definecolor{ansi-blue-intense}{HTML}{0065CA}
    \definecolor{ansi-magenta}{HTML}{D160C4}
    \definecolor{ansi-magenta-intense}{HTML}{A03196}
    \definecolor{ansi-cyan}{HTML}{60C6C8}
    \definecolor{ansi-cyan-intense}{HTML}{258F8F}
    \definecolor{ansi-white}{HTML}{C5C1B4}
    \definecolor{ansi-white-intense}{HTML}{A1A6B2}

    % commands and environments needed by pandoc snippets
    % extracted from the output of `pandoc -s`
    \providecommand{\tightlist}{%
      \setlength{\itemsep}{0pt}\setlength{\parskip}{0pt}}
    \DefineVerbatimEnvironment{Highlighting}{Verbatim}{commandchars=\\\{\}}
    % Add ',fontsize=\small' for more characters per line
    \newenvironment{Shaded}{}{}
    \newcommand{\KeywordTok}[1]{\textcolor[rgb]{0.00,0.44,0.13}{\textbf{{#1}}}}
    \newcommand{\DataTypeTok}[1]{\textcolor[rgb]{0.56,0.13,0.00}{{#1}}}
    \newcommand{\DecValTok}[1]{\textcolor[rgb]{0.25,0.63,0.44}{{#1}}}
    \newcommand{\BaseNTok}[1]{\textcolor[rgb]{0.25,0.63,0.44}{{#1}}}
    \newcommand{\FloatTok}[1]{\textcolor[rgb]{0.25,0.63,0.44}{{#1}}}
    \newcommand{\CharTok}[1]{\textcolor[rgb]{0.25,0.44,0.63}{{#1}}}
    \newcommand{\StringTok}[1]{\textcolor[rgb]{0.25,0.44,0.63}{{#1}}}
    \newcommand{\CommentTok}[1]{\textcolor[rgb]{0.38,0.63,0.69}{\textit{{#1}}}}
    \newcommand{\OtherTok}[1]{\textcolor[rgb]{0.00,0.44,0.13}{{#1}}}
    \newcommand{\AlertTok}[1]{\textcolor[rgb]{1.00,0.00,0.00}{\textbf{{#1}}}}
    \newcommand{\FunctionTok}[1]{\textcolor[rgb]{0.02,0.16,0.49}{{#1}}}
    \newcommand{\RegionMarkerTok}[1]{{#1}}
    \newcommand{\ErrorTok}[1]{\textcolor[rgb]{1.00,0.00,0.00}{\textbf{{#1}}}}
    \newcommand{\NormalTok}[1]{{#1}}
    
    % Additional commands for more recent versions of Pandoc
    \newcommand{\ConstantTok}[1]{\textcolor[rgb]{0.53,0.00,0.00}{{#1}}}
    \newcommand{\SpecialCharTok}[1]{\textcolor[rgb]{0.25,0.44,0.63}{{#1}}}
    \newcommand{\VerbatimStringTok}[1]{\textcolor[rgb]{0.25,0.44,0.63}{{#1}}}
    \newcommand{\SpecialStringTok}[1]{\textcolor[rgb]{0.73,0.40,0.53}{{#1}}}
    \newcommand{\ImportTok}[1]{{#1}}
    \newcommand{\DocumentationTok}[1]{\textcolor[rgb]{0.73,0.13,0.13}{\textit{{#1}}}}
    \newcommand{\AnnotationTok}[1]{\textcolor[rgb]{0.38,0.63,0.69}{\textbf{\textit{{#1}}}}}
    \newcommand{\CommentVarTok}[1]{\textcolor[rgb]{0.38,0.63,0.69}{\textbf{\textit{{#1}}}}}
    \newcommand{\VariableTok}[1]{\textcolor[rgb]{0.10,0.09,0.49}{{#1}}}
    \newcommand{\ControlFlowTok}[1]{\textcolor[rgb]{0.00,0.44,0.13}{\textbf{{#1}}}}
    \newcommand{\OperatorTok}[1]{\textcolor[rgb]{0.40,0.40,0.40}{{#1}}}
    \newcommand{\BuiltInTok}[1]{{#1}}
    \newcommand{\ExtensionTok}[1]{{#1}}
    \newcommand{\PreprocessorTok}[1]{\textcolor[rgb]{0.74,0.48,0.00}{{#1}}}
    \newcommand{\AttributeTok}[1]{\textcolor[rgb]{0.49,0.56,0.16}{{#1}}}
    \newcommand{\InformationTok}[1]{\textcolor[rgb]{0.38,0.63,0.69}{\textbf{\textit{{#1}}}}}
    \newcommand{\WarningTok}[1]{\textcolor[rgb]{0.38,0.63,0.69}{\textbf{\textit{{#1}}}}}
    
    
    % Define a nice break command that doesn't care if a line doesn't already
    % exist.
    \def\br{\hspace*{\fill} \\* }
    % Math Jax compatability definitions
    \def\gt{>}
    \def\lt{<}
    % Document parameters
    \title{P6}
    
    
    

    % Pygments definitions
    
\makeatletter
\def\PY@reset{\let\PY@it=\relax \let\PY@bf=\relax%
    \let\PY@ul=\relax \let\PY@tc=\relax%
    \let\PY@bc=\relax \let\PY@ff=\relax}
\def\PY@tok#1{\csname PY@tok@#1\endcsname}
\def\PY@toks#1+{\ifx\relax#1\empty\else%
    \PY@tok{#1}\expandafter\PY@toks\fi}
\def\PY@do#1{\PY@bc{\PY@tc{\PY@ul{%
    \PY@it{\PY@bf{\PY@ff{#1}}}}}}}
\def\PY#1#2{\PY@reset\PY@toks#1+\relax+\PY@do{#2}}

\expandafter\def\csname PY@tok@w\endcsname{\def\PY@tc##1{\textcolor[rgb]{0.73,0.73,0.73}{##1}}}
\expandafter\def\csname PY@tok@c\endcsname{\let\PY@it=\textit\def\PY@tc##1{\textcolor[rgb]{0.25,0.50,0.50}{##1}}}
\expandafter\def\csname PY@tok@cp\endcsname{\def\PY@tc##1{\textcolor[rgb]{0.74,0.48,0.00}{##1}}}
\expandafter\def\csname PY@tok@k\endcsname{\let\PY@bf=\textbf\def\PY@tc##1{\textcolor[rgb]{0.00,0.50,0.00}{##1}}}
\expandafter\def\csname PY@tok@kp\endcsname{\def\PY@tc##1{\textcolor[rgb]{0.00,0.50,0.00}{##1}}}
\expandafter\def\csname PY@tok@kt\endcsname{\def\PY@tc##1{\textcolor[rgb]{0.69,0.00,0.25}{##1}}}
\expandafter\def\csname PY@tok@o\endcsname{\def\PY@tc##1{\textcolor[rgb]{0.40,0.40,0.40}{##1}}}
\expandafter\def\csname PY@tok@ow\endcsname{\let\PY@bf=\textbf\def\PY@tc##1{\textcolor[rgb]{0.67,0.13,1.00}{##1}}}
\expandafter\def\csname PY@tok@nb\endcsname{\def\PY@tc##1{\textcolor[rgb]{0.00,0.50,0.00}{##1}}}
\expandafter\def\csname PY@tok@nf\endcsname{\def\PY@tc##1{\textcolor[rgb]{0.00,0.00,1.00}{##1}}}
\expandafter\def\csname PY@tok@nc\endcsname{\let\PY@bf=\textbf\def\PY@tc##1{\textcolor[rgb]{0.00,0.00,1.00}{##1}}}
\expandafter\def\csname PY@tok@nn\endcsname{\let\PY@bf=\textbf\def\PY@tc##1{\textcolor[rgb]{0.00,0.00,1.00}{##1}}}
\expandafter\def\csname PY@tok@ne\endcsname{\let\PY@bf=\textbf\def\PY@tc##1{\textcolor[rgb]{0.82,0.25,0.23}{##1}}}
\expandafter\def\csname PY@tok@nv\endcsname{\def\PY@tc##1{\textcolor[rgb]{0.10,0.09,0.49}{##1}}}
\expandafter\def\csname PY@tok@no\endcsname{\def\PY@tc##1{\textcolor[rgb]{0.53,0.00,0.00}{##1}}}
\expandafter\def\csname PY@tok@nl\endcsname{\def\PY@tc##1{\textcolor[rgb]{0.63,0.63,0.00}{##1}}}
\expandafter\def\csname PY@tok@ni\endcsname{\let\PY@bf=\textbf\def\PY@tc##1{\textcolor[rgb]{0.60,0.60,0.60}{##1}}}
\expandafter\def\csname PY@tok@na\endcsname{\def\PY@tc##1{\textcolor[rgb]{0.49,0.56,0.16}{##1}}}
\expandafter\def\csname PY@tok@nt\endcsname{\let\PY@bf=\textbf\def\PY@tc##1{\textcolor[rgb]{0.00,0.50,0.00}{##1}}}
\expandafter\def\csname PY@tok@nd\endcsname{\def\PY@tc##1{\textcolor[rgb]{0.67,0.13,1.00}{##1}}}
\expandafter\def\csname PY@tok@s\endcsname{\def\PY@tc##1{\textcolor[rgb]{0.73,0.13,0.13}{##1}}}
\expandafter\def\csname PY@tok@sd\endcsname{\let\PY@it=\textit\def\PY@tc##1{\textcolor[rgb]{0.73,0.13,0.13}{##1}}}
\expandafter\def\csname PY@tok@si\endcsname{\let\PY@bf=\textbf\def\PY@tc##1{\textcolor[rgb]{0.73,0.40,0.53}{##1}}}
\expandafter\def\csname PY@tok@se\endcsname{\let\PY@bf=\textbf\def\PY@tc##1{\textcolor[rgb]{0.73,0.40,0.13}{##1}}}
\expandafter\def\csname PY@tok@sr\endcsname{\def\PY@tc##1{\textcolor[rgb]{0.73,0.40,0.53}{##1}}}
\expandafter\def\csname PY@tok@ss\endcsname{\def\PY@tc##1{\textcolor[rgb]{0.10,0.09,0.49}{##1}}}
\expandafter\def\csname PY@tok@sx\endcsname{\def\PY@tc##1{\textcolor[rgb]{0.00,0.50,0.00}{##1}}}
\expandafter\def\csname PY@tok@m\endcsname{\def\PY@tc##1{\textcolor[rgb]{0.40,0.40,0.40}{##1}}}
\expandafter\def\csname PY@tok@gh\endcsname{\let\PY@bf=\textbf\def\PY@tc##1{\textcolor[rgb]{0.00,0.00,0.50}{##1}}}
\expandafter\def\csname PY@tok@gu\endcsname{\let\PY@bf=\textbf\def\PY@tc##1{\textcolor[rgb]{0.50,0.00,0.50}{##1}}}
\expandafter\def\csname PY@tok@gd\endcsname{\def\PY@tc##1{\textcolor[rgb]{0.63,0.00,0.00}{##1}}}
\expandafter\def\csname PY@tok@gi\endcsname{\def\PY@tc##1{\textcolor[rgb]{0.00,0.63,0.00}{##1}}}
\expandafter\def\csname PY@tok@gr\endcsname{\def\PY@tc##1{\textcolor[rgb]{1.00,0.00,0.00}{##1}}}
\expandafter\def\csname PY@tok@ge\endcsname{\let\PY@it=\textit}
\expandafter\def\csname PY@tok@gs\endcsname{\let\PY@bf=\textbf}
\expandafter\def\csname PY@tok@gp\endcsname{\let\PY@bf=\textbf\def\PY@tc##1{\textcolor[rgb]{0.00,0.00,0.50}{##1}}}
\expandafter\def\csname PY@tok@go\endcsname{\def\PY@tc##1{\textcolor[rgb]{0.53,0.53,0.53}{##1}}}
\expandafter\def\csname PY@tok@gt\endcsname{\def\PY@tc##1{\textcolor[rgb]{0.00,0.27,0.87}{##1}}}
\expandafter\def\csname PY@tok@err\endcsname{\def\PY@bc##1{\setlength{\fboxsep}{0pt}\fcolorbox[rgb]{1.00,0.00,0.00}{1,1,1}{\strut ##1}}}
\expandafter\def\csname PY@tok@kc\endcsname{\let\PY@bf=\textbf\def\PY@tc##1{\textcolor[rgb]{0.00,0.50,0.00}{##1}}}
\expandafter\def\csname PY@tok@kd\endcsname{\let\PY@bf=\textbf\def\PY@tc##1{\textcolor[rgb]{0.00,0.50,0.00}{##1}}}
\expandafter\def\csname PY@tok@kn\endcsname{\let\PY@bf=\textbf\def\PY@tc##1{\textcolor[rgb]{0.00,0.50,0.00}{##1}}}
\expandafter\def\csname PY@tok@kr\endcsname{\let\PY@bf=\textbf\def\PY@tc##1{\textcolor[rgb]{0.00,0.50,0.00}{##1}}}
\expandafter\def\csname PY@tok@bp\endcsname{\def\PY@tc##1{\textcolor[rgb]{0.00,0.50,0.00}{##1}}}
\expandafter\def\csname PY@tok@fm\endcsname{\def\PY@tc##1{\textcolor[rgb]{0.00,0.00,1.00}{##1}}}
\expandafter\def\csname PY@tok@vc\endcsname{\def\PY@tc##1{\textcolor[rgb]{0.10,0.09,0.49}{##1}}}
\expandafter\def\csname PY@tok@vg\endcsname{\def\PY@tc##1{\textcolor[rgb]{0.10,0.09,0.49}{##1}}}
\expandafter\def\csname PY@tok@vi\endcsname{\def\PY@tc##1{\textcolor[rgb]{0.10,0.09,0.49}{##1}}}
\expandafter\def\csname PY@tok@vm\endcsname{\def\PY@tc##1{\textcolor[rgb]{0.10,0.09,0.49}{##1}}}
\expandafter\def\csname PY@tok@sa\endcsname{\def\PY@tc##1{\textcolor[rgb]{0.73,0.13,0.13}{##1}}}
\expandafter\def\csname PY@tok@sb\endcsname{\def\PY@tc##1{\textcolor[rgb]{0.73,0.13,0.13}{##1}}}
\expandafter\def\csname PY@tok@sc\endcsname{\def\PY@tc##1{\textcolor[rgb]{0.73,0.13,0.13}{##1}}}
\expandafter\def\csname PY@tok@dl\endcsname{\def\PY@tc##1{\textcolor[rgb]{0.73,0.13,0.13}{##1}}}
\expandafter\def\csname PY@tok@s2\endcsname{\def\PY@tc##1{\textcolor[rgb]{0.73,0.13,0.13}{##1}}}
\expandafter\def\csname PY@tok@sh\endcsname{\def\PY@tc##1{\textcolor[rgb]{0.73,0.13,0.13}{##1}}}
\expandafter\def\csname PY@tok@s1\endcsname{\def\PY@tc##1{\textcolor[rgb]{0.73,0.13,0.13}{##1}}}
\expandafter\def\csname PY@tok@mb\endcsname{\def\PY@tc##1{\textcolor[rgb]{0.40,0.40,0.40}{##1}}}
\expandafter\def\csname PY@tok@mf\endcsname{\def\PY@tc##1{\textcolor[rgb]{0.40,0.40,0.40}{##1}}}
\expandafter\def\csname PY@tok@mh\endcsname{\def\PY@tc##1{\textcolor[rgb]{0.40,0.40,0.40}{##1}}}
\expandafter\def\csname PY@tok@mi\endcsname{\def\PY@tc##1{\textcolor[rgb]{0.40,0.40,0.40}{##1}}}
\expandafter\def\csname PY@tok@il\endcsname{\def\PY@tc##1{\textcolor[rgb]{0.40,0.40,0.40}{##1}}}
\expandafter\def\csname PY@tok@mo\endcsname{\def\PY@tc##1{\textcolor[rgb]{0.40,0.40,0.40}{##1}}}
\expandafter\def\csname PY@tok@ch\endcsname{\let\PY@it=\textit\def\PY@tc##1{\textcolor[rgb]{0.25,0.50,0.50}{##1}}}
\expandafter\def\csname PY@tok@cm\endcsname{\let\PY@it=\textit\def\PY@tc##1{\textcolor[rgb]{0.25,0.50,0.50}{##1}}}
\expandafter\def\csname PY@tok@cpf\endcsname{\let\PY@it=\textit\def\PY@tc##1{\textcolor[rgb]{0.25,0.50,0.50}{##1}}}
\expandafter\def\csname PY@tok@c1\endcsname{\let\PY@it=\textit\def\PY@tc##1{\textcolor[rgb]{0.25,0.50,0.50}{##1}}}
\expandafter\def\csname PY@tok@cs\endcsname{\let\PY@it=\textit\def\PY@tc##1{\textcolor[rgb]{0.25,0.50,0.50}{##1}}}

\def\PYZbs{\char`\\}
\def\PYZus{\char`\_}
\def\PYZob{\char`\{}
\def\PYZcb{\char`\}}
\def\PYZca{\char`\^}
\def\PYZam{\char`\&}
\def\PYZlt{\char`\<}
\def\PYZgt{\char`\>}
\def\PYZsh{\char`\#}
\def\PYZpc{\char`\%}
\def\PYZdl{\char`\$}
\def\PYZhy{\char`\-}
\def\PYZsq{\char`\'}
\def\PYZdq{\char`\"}
\def\PYZti{\char`\~}
% for compatibility with earlier versions
\def\PYZat{@}
\def\PYZlb{[}
\def\PYZrb{]}
\makeatother


    % Exact colors from NB
    \definecolor{incolor}{rgb}{0.0, 0.0, 0.5}
    \definecolor{outcolor}{rgb}{0.545, 0.0, 0.0}



    
    % Prevent overflowing lines due to hard-to-break entities
    \sloppy 
    % Setup hyperref package
    \hypersetup{
      breaklinks=true,  % so long urls are correctly broken across lines
      colorlinks=true,
      urlcolor=urlcolor,
      linkcolor=linkcolor,
      citecolor=citecolor,
      }
    % Slightly bigger margins than the latex defaults
    
    \geometry{verbose,tmargin=1in,bmargin=1in,lmargin=1in,rmargin=1in}
    
    

    \begin{document}
    
    
    \maketitle
    
    

    
    \section{Practica 6 Aprendizaje Automático y Minería de
Datos}\label{practica-6-aprendizaje-automuxe1tico-y-mineruxeda-de-datos}

Utilización de Support Vector Machines para casos básicos de
clasificación y Filtrado de correos spam mediante el uso de SVMs. Por
Mario Jimenez y Manuel Hernández

    \subsection{Importado de librerías}\label{importado-de-libreruxedas}

    \begin{Verbatim}[commandchars=\\\{\}]
{\color{incolor}In [{\color{incolor}218}]:} \PY{k+kn}{import} \PY{n+nn}{numpy} \PY{k}{as} \PY{n+nn}{np}
          \PY{k+kn}{from} \PY{n+nn}{scipy}\PY{n+nn}{.}\PY{n+nn}{io} \PY{k}{import} \PY{n}{loadmat}
          \PY{k+kn}{import} \PY{n+nn}{matplotlib}\PY{n+nn}{.}\PY{n+nn}{pyplot} \PY{k}{as} \PY{n+nn}{plt}
          \PY{k+kn}{from} \PY{n+nn}{sklearn} \PY{k}{import} \PY{n}{svm}
          \PY{k+kn}{from} \PY{n+nn}{mpl\PYZus{}toolkits}\PY{n+nn}{.}\PY{n+nn}{mplot3d} \PY{k}{import} \PY{n}{Axes3D}
\end{Verbatim}


    \subsection{Conjunto de datos 1: Kernel
Lineal}\label{conjunto-de-datos-1-kernel-lineal}

    \begin{Verbatim}[commandchars=\\\{\}]
{\color{incolor}In [{\color{incolor}219}]:} \PY{n}{dataset1} \PY{o}{=} \PY{n}{loadmat}\PY{p}{(}\PY{l+s+s2}{\PYZdq{}}\PY{l+s+s2}{ex6data1.mat}\PY{l+s+s2}{\PYZdq{}}\PY{p}{)}
          \PY{n}{X\PYZus{}1} \PY{o}{=} \PY{n}{dataset1}\PY{p}{[}\PY{l+s+s1}{\PYZsq{}}\PY{l+s+s1}{X}\PY{l+s+s1}{\PYZsq{}}\PY{p}{]}
          \PY{n}{Y\PYZus{}1} \PY{o}{=} \PY{n}{dataset1}\PY{p}{[}\PY{l+s+s1}{\PYZsq{}}\PY{l+s+s1}{y}\PY{l+s+s1}{\PYZsq{}}\PY{p}{]}
          
          \PY{n}{pos} \PY{o}{=} \PY{n}{np}\PY{o}{.}\PY{n}{array}\PY{p}{(}\PY{p}{[}\PY{n}{X\PYZus{}1}\PY{p}{[}\PY{n}{i}\PY{p}{]} \PY{k}{for} \PY{n}{i} \PY{o+ow}{in} \PY{n+nb}{range}\PY{p}{(}\PY{n+nb}{len}\PY{p}{(}\PY{n}{X\PYZus{}1}\PY{p}{)}\PY{p}{)}\PY{k}{if} \PY{n}{Y\PYZus{}1}\PY{p}{[}\PY{n}{i}\PY{p}{]} \PY{o}{==} \PY{l+m+mi}{1}\PY{p}{]}\PY{p}{)}
          \PY{n}{neg} \PY{o}{=} \PY{n}{np}\PY{o}{.}\PY{n}{array}\PY{p}{(}\PY{p}{[}\PY{n}{X\PYZus{}1}\PY{p}{[}\PY{n}{i}\PY{p}{]} \PY{k}{for} \PY{n}{i} \PY{o+ow}{in} \PY{n+nb}{range}\PY{p}{(}\PY{n+nb}{len}\PY{p}{(}\PY{n}{X\PYZus{}1}\PY{p}{)}\PY{p}{)}\PY{k}{if} \PY{n}{Y\PYZus{}1}\PY{p}{[}\PY{n}{i}\PY{p}{]} \PY{o}{==} \PY{l+m+mi}{0}\PY{p}{]}\PY{p}{)}
\end{Verbatim}


    \subsubsection{Inicialización del kernel
lineal}\label{inicializaciuxf3n-del-kernel-lineal}

    \begin{Verbatim}[commandchars=\\\{\}]
{\color{incolor}In [{\color{incolor}220}]:} \PY{n}{kernel\PYZus{}lin} \PY{o}{=} \PY{n}{svm}\PY{o}{.}\PY{n}{SVC}\PY{p}{(}\PY{n}{C}\PY{o}{=} \PY{l+m+mi}{1}\PY{p}{,} \PY{n}{kernel} \PY{o}{=} \PY{l+s+s1}{\PYZsq{}}\PY{l+s+s1}{linear}\PY{l+s+s1}{\PYZsq{}}\PY{p}{)}
          \PY{n}{res} \PY{o}{=} \PY{n}{kernel\PYZus{}lin}\PY{o}{.}\PY{n}{fit}\PY{p}{(}\PY{n}{X\PYZus{}1}\PY{p}{,} \PY{n}{Y\PYZus{}1}\PY{o}{.}\PY{n}{flatten}\PY{p}{(}\PY{p}{)}\PY{p}{)}
          \PY{n}{pred} \PY{o}{=} \PY{n}{kernel\PYZus{}lin}\PY{o}{.}\PY{n}{predict}\PY{p}{(}\PY{n}{np}\PY{o}{.}\PY{n}{array}\PY{p}{(}\PY{p}{[}\PY{p}{[}\PY{l+m+mi}{2}\PY{p}{,} \PY{l+m+mi}{4}\PY{p}{]}\PY{p}{]}\PY{p}{)}\PY{p}{)}
          
          \PY{n+nb}{print}\PY{p}{(}\PY{n}{res}\PY{p}{,} \PY{l+s+s2}{\PYZdq{}}\PY{l+s+se}{\PYZbs{}n}\PY{l+s+s2}{\PYZdq{}}\PY{p}{)}
          \PY{n+nb}{print}\PY{p}{(}\PY{l+s+s2}{\PYZdq{}}\PY{l+s+s2}{Valor predicho para (2, 4): }\PY{l+s+s2}{\PYZdq{}}\PY{p}{,} \PY{n}{pred}\PY{p}{)}
\end{Verbatim}


    \begin{Verbatim}[commandchars=\\\{\}]
SVC(C=1, cache\_size=200, class\_weight=None, coef0=0.0,
  decision\_function\_shape='ovr', degree=3, gamma='auto', kernel='linear',
  max\_iter=-1, probability=False, random\_state=None, shrinking=True,
  tol=0.001, verbose=False) 

Valor predicho para (2, 4):  [1]

    \end{Verbatim}

    \subsubsection{Representación de
datos}\label{representaciuxf3n-de-datos}

    \begin{Verbatim}[commandchars=\\\{\}]
{\color{incolor}In [{\color{incolor}221}]:} \PY{k}{def} \PY{n+nf}{dibujaDatos}\PY{p}{(}\PY{n}{pos}\PY{p}{,} \PY{n}{neg}\PY{p}{)}\PY{p}{:}
              \PY{n}{plt}\PY{o}{.}\PY{n}{plot}\PY{p}{(}\PY{n}{pos}\PY{p}{[}\PY{p}{:}\PY{p}{,} \PY{l+m+mi}{0}\PY{p}{]}\PY{p}{,} \PY{n}{pos} \PY{p}{[}\PY{p}{:}\PY{p}{,} \PY{l+m+mi}{1}\PY{p}{]}\PY{p}{,} \PY{l+s+s1}{\PYZsq{}}\PY{l+s+s1}{gx}\PY{l+s+s1}{\PYZsq{}}\PY{p}{,} \PY{n}{label} \PY{o}{=} \PY{l+s+s1}{\PYZsq{}}\PY{l+s+s1}{valores positivos}\PY{l+s+s1}{\PYZsq{}}\PY{p}{)}
              \PY{n}{plt}\PY{o}{.}\PY{n}{plot}\PY{p}{(}\PY{n}{neg}\PY{p}{[}\PY{p}{:}\PY{p}{,} \PY{l+m+mi}{0}\PY{p}{]}\PY{p}{,} \PY{n}{neg} \PY{p}{[}\PY{p}{:}\PY{p}{,} \PY{l+m+mi}{1}\PY{p}{]}\PY{p}{,} \PY{l+s+s1}{\PYZsq{}}\PY{l+s+s1}{ro}\PY{l+s+s1}{\PYZsq{}}\PY{p}{,} \PY{n}{label} \PY{o}{=} \PY{l+s+s1}{\PYZsq{}}\PY{l+s+s1}{valores negativos}\PY{l+s+s1}{\PYZsq{}}\PY{p}{)}
              \PY{n}{plt}\PY{o}{.}\PY{n}{xlabel}\PY{p}{(}\PY{l+s+s1}{\PYZsq{}}\PY{l+s+s1}{Valores eje X}\PY{l+s+s1}{\PYZsq{}}\PY{p}{)}
              \PY{n}{plt}\PY{o}{.}\PY{n}{ylabel}\PY{p}{(}\PY{l+s+s1}{\PYZsq{}}\PY{l+s+s1}{Valores eje Y}\PY{l+s+s1}{\PYZsq{}}\PY{p}{)}
              \PY{n}{plt}\PY{o}{.}\PY{n}{legend}\PY{p}{(}\PY{p}{)}
              \PY{n}{plt}\PY{o}{.}\PY{n}{grid}\PY{p}{(}\PY{k+kc}{True}\PY{p}{)}
          
          \PY{k}{def} \PY{n+nf}{dibujaLimite}\PY{p}{(}\PY{n}{svm}\PY{p}{,} \PY{n}{pos}\PY{p}{,} \PY{n}{neg}\PY{p}{,} \PY{n}{minx}\PY{p}{,} \PY{n}{maxx}\PY{p}{,} \PY{n}{miny}\PY{p}{,} \PY{n}{maxy}\PY{p}{)}\PY{p}{:}
              \PY{n}{figure} \PY{o}{=} \PY{n}{plt}\PY{o}{.}\PY{n}{figure}\PY{p}{(}\PY{p}{)}
              \PY{n}{X} \PY{o}{=} \PY{n}{np}\PY{o}{.}\PY{n}{linspace}\PY{p}{(}\PY{n}{minx}\PY{p}{,} \PY{n}{maxx}\PY{p}{,} \PY{l+m+mi}{200}\PY{p}{)}
              \PY{n}{Y} \PY{o}{=} \PY{n}{np}\PY{o}{.}\PY{n}{linspace}\PY{p}{(}\PY{n}{miny}\PY{p}{,} \PY{n}{maxy}\PY{p}{,} \PY{l+m+mi}{200}\PY{p}{)}  
              \PY{n}{zvals} \PY{o}{=} \PY{n}{np}\PY{o}{.}\PY{n}{zeros}\PY{p}{(}\PY{n}{shape} \PY{o}{=} \PY{p}{(}\PY{n+nb}{len}\PY{p}{(}\PY{n}{X}\PY{p}{)}\PY{p}{,} \PY{n+nb}{len}\PY{p}{(}\PY{n}{Y}\PY{p}{)}\PY{p}{)}\PY{p}{)}
              
              \PY{k}{for} \PY{n}{i} \PY{o+ow}{in} \PY{n+nb}{range}\PY{p}{(}\PY{n}{X}\PY{o}{.}\PY{n}{shape}\PY{p}{[}\PY{l+m+mi}{0}\PY{p}{]}\PY{p}{)}\PY{p}{:}
                  \PY{k}{for} \PY{n}{j} \PY{o+ow}{in} \PY{n+nb}{range}\PY{p}{(}\PY{n}{Y}\PY{o}{.}\PY{n}{shape}\PY{p}{[}\PY{l+m+mi}{0}\PY{p}{]}\PY{p}{)}\PY{p}{:}
                      \PY{n}{aux} \PY{o}{=} \PY{n}{np}\PY{o}{.}\PY{n}{array}\PY{p}{(}\PY{p}{[}\PY{p}{[}\PY{n}{X}\PY{p}{[}\PY{n}{i}\PY{p}{]}\PY{p}{,}\PY{n}{Y}\PY{p}{[}\PY{n}{j}\PY{p}{]}\PY{p}{]}\PY{p}{]}\PY{p}{)}
                      \PY{n}{zvals}\PY{p}{[}\PY{n}{i}\PY{p}{]}\PY{p}{[}\PY{n}{j}\PY{p}{]} \PY{o}{=} \PY{n+nb}{float}\PY{p}{(}\PY{n}{svm}\PY{o}{.}\PY{n}{predict}\PY{p}{(}\PY{n}{aux}\PY{p}{)}\PY{p}{)}
               
              \PY{n}{zvals} \PY{o}{=} \PY{n}{zvals}\PY{o}{.}\PY{n}{T}  
              \PY{n}{dibujaDatos}\PY{p}{(}\PY{n}{pos}\PY{p}{,} \PY{n}{neg}\PY{p}{)}
              \PY{n}{a}\PY{p}{,} \PY{n}{b} \PY{o}{=} \PY{n}{np}\PY{o}{.}\PY{n}{meshgrid} \PY{p}{(}\PY{n}{X}\PY{p}{,} \PY{n}{Y}\PY{p}{)}
              \PY{n}{contour} \PY{o}{=} \PY{n}{plt}\PY{o}{.}\PY{n}{contour} \PY{p}{(}\PY{n}{X}\PY{p}{,} \PY{n}{Y}\PY{p}{,} \PY{n}{zvals}\PY{p}{)}    
\end{Verbatim}


    En los siguientes apartados podemos ver la diferencia, para el conjunto
de datos proporcionado, de utilizar un coeficiente de penalización de C
= 1 contra C = 100.

    \paragraph{C = 1.0}\label{c-1.0}

    \begin{Verbatim}[commandchars=\\\{\}]
{\color{incolor}In [{\color{incolor}222}]:} \PY{n}{plt}\PY{o}{.}\PY{n}{figure}\PY{p}{(}\PY{p}{)}
          \PY{n}{dibujaLimite}\PY{p}{(}\PY{n}{kernel\PYZus{}lin}\PY{p}{,} \PY{n}{pos}\PY{p}{,} \PY{n}{neg}\PY{p}{,} \PY{o}{\PYZhy{}}\PY{l+m+mi}{1}\PY{p}{,}\PY{l+m+mi}{5}\PY{p}{,}\PY{l+m+mi}{1}\PY{p}{,}\PY{l+m+mi}{5}\PY{p}{)}
          \PY{n}{plt}\PY{o}{.}\PY{n}{show}\PY{p}{(}\PY{p}{)}
\end{Verbatim}


    
    \begin{verbatim}
<Figure size 432x288 with 0 Axes>
    \end{verbatim}

    
    \begin{center}
    \adjustimage{max size={0.9\linewidth}{0.9\paperheight}}{output_11_1.png}
    \end{center}
    { \hspace*{\fill} \\}
    
    \paragraph{C = 100}\label{c-100}

    \begin{Verbatim}[commandchars=\\\{\}]
{\color{incolor}In [{\color{incolor}223}]:} \PY{n}{kernel\PYZus{}lin} \PY{o}{=} \PY{n}{svm}\PY{o}{.}\PY{n}{SVC}\PY{p}{(}\PY{n}{C}\PY{o}{=} \PY{l+m+mi}{100}\PY{p}{,} \PY{n}{kernel} \PY{o}{=} \PY{l+s+s1}{\PYZsq{}}\PY{l+s+s1}{linear}\PY{l+s+s1}{\PYZsq{}}\PY{p}{)}
          \PY{n}{kernel\PYZus{}lin}\PY{o}{.}\PY{n}{fit}\PY{p}{(}\PY{n}{X\PYZus{}1}\PY{p}{,} \PY{n}{Y\PYZus{}1}\PY{o}{.}\PY{n}{flatten}\PY{p}{(}\PY{p}{)}\PY{p}{)}
          
          \PY{n}{plt}\PY{o}{.}\PY{n}{figure}\PY{p}{(}\PY{p}{)}
          \PY{n}{dibujaLimite}\PY{p}{(}\PY{n}{kernel\PYZus{}lin}\PY{p}{,} \PY{n}{pos}\PY{p}{,} \PY{n}{neg}\PY{p}{,} \PY{o}{\PYZhy{}}\PY{l+m+mi}{1}\PY{p}{,}\PY{l+m+mi}{5}\PY{p}{,}\PY{l+m+mi}{1}\PY{p}{,}\PY{l+m+mi}{5}\PY{p}{)}
          \PY{n}{plt}\PY{o}{.}\PY{n}{show}\PY{p}{(}\PY{p}{)}
\end{Verbatim}


    
    \begin{verbatim}
<Figure size 432x288 with 0 Axes>
    \end{verbatim}

    
    \begin{center}
    \adjustimage{max size={0.9\linewidth}{0.9\paperheight}}{output_13_1.png}
    \end{center}
    { \hspace*{\fill} \\}
    
    \subsection{Conjunto de datos 2: Kernel
Gaussiano}\label{conjunto-de-datos-2-kernel-gaussiano}

    \subsubsection{Obtención de datos}\label{obtenciuxf3n-de-datos}

    \begin{Verbatim}[commandchars=\\\{\}]
{\color{incolor}In [{\color{incolor}224}]:} \PY{n}{dataset2} \PY{o}{=} \PY{n}{loadmat}\PY{p}{(}\PY{l+s+s2}{\PYZdq{}}\PY{l+s+s2}{ex6data2.mat}\PY{l+s+s2}{\PYZdq{}}\PY{p}{)}
          \PY{n}{X\PYZus{}2} \PY{o}{=} \PY{n}{dataset2}\PY{p}{[}\PY{l+s+s1}{\PYZsq{}}\PY{l+s+s1}{X}\PY{l+s+s1}{\PYZsq{}}\PY{p}{]}
          \PY{n}{Y\PYZus{}2} \PY{o}{=} \PY{n}{dataset2}\PY{p}{[}\PY{l+s+s1}{\PYZsq{}}\PY{l+s+s1}{y}\PY{l+s+s1}{\PYZsq{}}\PY{p}{]}
          
          \PY{n}{pos} \PY{o}{=} \PY{n}{np}\PY{o}{.}\PY{n}{array}\PY{p}{(}\PY{p}{[}\PY{n}{X\PYZus{}2}\PY{p}{[}\PY{n}{i}\PY{p}{]} \PY{k}{for} \PY{n}{i} \PY{o+ow}{in} \PY{n+nb}{range}\PY{p}{(}\PY{n+nb}{len}\PY{p}{(}\PY{n}{X\PYZus{}2}\PY{p}{)}\PY{p}{)}\PY{k}{if} \PY{n}{Y\PYZus{}2}\PY{p}{[}\PY{n}{i}\PY{p}{]} \PY{o}{==} \PY{l+m+mi}{1}\PY{p}{]}\PY{p}{)}
          \PY{n}{neg} \PY{o}{=} \PY{n}{np}\PY{o}{.}\PY{n}{array}\PY{p}{(}\PY{p}{[}\PY{n}{X\PYZus{}2}\PY{p}{[}\PY{n}{i}\PY{p}{]} \PY{k}{for} \PY{n}{i} \PY{o+ow}{in} \PY{n+nb}{range}\PY{p}{(}\PY{n+nb}{len}\PY{p}{(}\PY{n}{X\PYZus{}2}\PY{p}{)}\PY{p}{)}\PY{k}{if} \PY{n}{Y\PYZus{}2}\PY{p}{[}\PY{n}{i}\PY{p}{]} \PY{o}{==} \PY{l+m+mi}{0}\PY{p}{]}\PY{p}{)}
          
          
          \PY{n}{plt}\PY{o}{.}\PY{n}{figure}\PY{p}{(}\PY{p}{)}
          \PY{n}{dibujaDatos}\PY{p}{(}\PY{n}{pos}\PY{p}{,} \PY{n}{neg}\PY{p}{)}
          \PY{n}{plt}\PY{o}{.}\PY{n}{show}\PY{p}{(}\PY{p}{)}
\end{Verbatim}


    \begin{center}
    \adjustimage{max size={0.9\linewidth}{0.9\paperheight}}{output_16_0.png}
    \end{center}
    { \hspace*{\fill} \\}
    
    \subsubsection{Entrenamiento del SVM}\label{entrenamiento-del-svm}

    El SVM gaussiano recibe 2 parámetros que podemos ajustar: - C:
Penalización a elementos mal clasificados - gamma: Influencia de cada
ejemplo de entrenamiento en el resultado

    \begin{Verbatim}[commandchars=\\\{\}]
{\color{incolor}In [{\color{incolor}225}]:} \PY{n}{sigma} \PY{o}{=} \PY{l+m+mf}{0.1}
          \PY{n}{kernel\PYZus{}gaussiano} \PY{o}{=} \PY{n}{svm}\PY{o}{.}\PY{n}{SVC}\PY{p}{(}\PY{n}{C} \PY{o}{=} \PY{l+m+mi}{1}\PY{p}{,} \PY{n}{kernel} \PY{o}{=} \PY{l+s+s1}{\PYZsq{}}\PY{l+s+s1}{rbf}\PY{l+s+s1}{\PYZsq{}}\PY{p}{,} \PY{n}{gamma} \PY{o}{=} \PY{l+m+mi}{1}\PY{o}{/}\PY{p}{(}\PY{l+m+mi}{2}\PY{o}{*}\PY{n}{sigma} \PY{o}{*}\PY{o}{*}\PY{l+m+mi}{2}\PY{p}{)}\PY{p}{)}
          \PY{n}{kernel\PYZus{}gaussiano}\PY{o}{.}\PY{n}{fit}\PY{p}{(}\PY{n}{X\PYZus{}2}\PY{p}{,} \PY{n}{Y\PYZus{}2}\PY{o}{.}\PY{n}{flatten}\PY{p}{(}\PY{p}{)}\PY{p}{)}
          
          \PY{n}{dibujaLimite}\PY{p}{(}\PY{n}{kernel\PYZus{}gaussiano}\PY{p}{,} \PY{n}{pos}\PY{p}{,} \PY{n}{neg}\PY{p}{,} \PY{l+m+mf}{0.0}\PY{p}{,}\PY{l+m+mf}{1.0}\PY{p}{,}\PY{l+m+mf}{0.4}\PY{p}{,}\PY{l+m+mf}{1.0}\PY{p}{)}
\end{Verbatim}


    \begin{center}
    \adjustimage{max size={0.9\linewidth}{0.9\paperheight}}{output_19_0.png}
    \end{center}
    { \hspace*{\fill} \\}
    
    \subsection{Conjunto de datos 3: Evaluación del
modelo}\label{conjunto-de-datos-3-evaluaciuxf3n-del-modelo}

    En este apartado evaluaremos la precisión de un modelo mediante dos
conjunto de datos distintos (entrenamiento y evaluación) y el ajuste de
los parametros C y gamma.

    \subsubsection{Obtencion de datos}\label{obtencion-de-datos}

    \begin{Verbatim}[commandchars=\\\{\}]
{\color{incolor}In [{\color{incolor}238}]:} \PY{n}{dataset3} \PY{o}{=} \PY{n}{loadmat}\PY{p}{(}\PY{l+s+s2}{\PYZdq{}}\PY{l+s+s2}{ex6data3.mat}\PY{l+s+s2}{\PYZdq{}}\PY{p}{)}
          \PY{n}{X\PYZus{}3} \PY{o}{=} \PY{n}{dataset3}\PY{p}{[}\PY{l+s+s1}{\PYZsq{}}\PY{l+s+s1}{X}\PY{l+s+s1}{\PYZsq{}}\PY{p}{]}
          \PY{n}{Y\PYZus{}3} \PY{o}{=} \PY{n}{dataset3}\PY{p}{[}\PY{l+s+s1}{\PYZsq{}}\PY{l+s+s1}{y}\PY{l+s+s1}{\PYZsq{}}\PY{p}{]}
          \PY{n}{X\PYZus{}Test} \PY{o}{=} \PY{n}{dataset3}\PY{p}{[}\PY{l+s+s1}{\PYZsq{}}\PY{l+s+s1}{Xval}\PY{l+s+s1}{\PYZsq{}}\PY{p}{]}
          \PY{n}{Y\PYZus{}Test} \PY{o}{=} \PY{n}{dataset3}\PY{p}{[}\PY{l+s+s1}{\PYZsq{}}\PY{l+s+s1}{yval}\PY{l+s+s1}{\PYZsq{}}\PY{p}{]}
          
          \PY{n}{pos\PYZus{}3} \PY{o}{=} \PY{n}{np}\PY{o}{.}\PY{n}{array}\PY{p}{(}\PY{p}{[}\PY{n}{X\PYZus{}3}\PY{p}{[}\PY{n}{i}\PY{p}{]} \PY{k}{for} \PY{n}{i} \PY{o+ow}{in} \PY{n+nb}{range}\PY{p}{(}\PY{n+nb}{len}\PY{p}{(}\PY{n}{X\PYZus{}3}\PY{p}{)}\PY{p}{)}\PY{k}{if} \PY{n}{Y\PYZus{}3}\PY{p}{[}\PY{n}{i}\PY{p}{]} \PY{o}{==} \PY{l+m+mi}{1}\PY{p}{]}\PY{p}{)}
          \PY{n}{neg\PYZus{}3}\PY{o}{=} \PY{n}{np}\PY{o}{.}\PY{n}{array}\PY{p}{(}\PY{p}{[}\PY{n}{X\PYZus{}3}\PY{p}{[}\PY{n}{i}\PY{p}{]} \PY{k}{for} \PY{n}{i} \PY{o+ow}{in} \PY{n+nb}{range}\PY{p}{(}\PY{n+nb}{len}\PY{p}{(}\PY{n}{X\PYZus{}3}\PY{p}{)}\PY{p}{)}\PY{k}{if} \PY{n}{Y\PYZus{}3}\PY{p}{[}\PY{n}{i}\PY{p}{]} \PY{o}{==} \PY{l+m+mi}{0}\PY{p}{]}\PY{p}{)}
          
          \PY{n}{plt}\PY{o}{.}\PY{n}{figure}\PY{p}{(}\PY{p}{)}
          \PY{n}{dibujaDatos}\PY{p}{(}\PY{n}{pos\PYZus{}3}\PY{p}{,} \PY{n}{neg\PYZus{}3}\PY{p}{)}
          \PY{n}{plt}\PY{o}{.}\PY{n}{show}\PY{p}{(}\PY{p}{)}
\end{Verbatim}


    \begin{center}
    \adjustimage{max size={0.9\linewidth}{0.9\paperheight}}{output_23_0.png}
    \end{center}
    { \hspace*{\fill} \\}
    
    Trataremos ahora de obtener el modelo con menor porcentaje de fallos
sobre el conjunto de datos. Para ello debemos comprobar, una vez
entrenado, cuantos fallos y aciertos efectúa.

    \begin{Verbatim}[commandchars=\\\{\}]
{\color{incolor}In [{\color{incolor}237}]:} \PY{n}{Cvals} \PY{o}{=} \PY{n}{np}\PY{o}{.}\PY{n}{array}\PY{p}{(}\PY{p}{[}\PY{l+m+mf}{0.01}\PY{p}{,} \PY{l+m+mf}{0.03}\PY{p}{,} \PY{l+m+mf}{0.1}\PY{p}{,} \PY{l+m+mf}{0.3}\PY{p}{,} \PY{l+m+mf}{1.}\PY{p}{,} \PY{l+m+mf}{3.}\PY{p}{,} \PY{l+m+mf}{10.}\PY{p}{,} \PY{l+m+mf}{30.}\PY{p}{]}\PY{p}{)}
          \PY{n}{Sigmavals} \PY{o}{=} \PY{n}{np}\PY{o}{.}\PY{n}{copy}\PY{p}{(}\PY{n}{Cvals}\PY{p}{)}
          
          
          \PY{n}{bestC} \PY{o}{=} \PY{l+m+mf}{0.01}
          \PY{n}{bestSigma} \PY{o}{=} \PY{l+m+mf}{0.01}
          \PY{n}{bestScore} \PY{o}{=} \PY{o}{\PYZhy{}}\PY{l+m+mi}{1}
          
          \PY{k}{for} \PY{n}{C} \PY{o+ow}{in} \PY{n}{Cvals}\PY{p}{:}
              \PY{k}{for} \PY{n}{sigma} \PY{o+ow}{in} \PY{n}{Sigmavals}\PY{p}{:}
                  \PY{n}{gamma} \PY{o}{=} \PY{l+m+mi}{1}\PY{o}{/}\PY{p}{(}\PY{l+m+mi}{2}\PY{o}{*}\PY{n}{sigma} \PY{o}{*}\PY{o}{*}\PY{l+m+mi}{2}\PY{p}{)}
                  \PY{n}{aux\PYZus{}kernel} \PY{o}{=} \PY{n}{svm}\PY{o}{.}\PY{n}{SVC}\PY{p}{(}\PY{n}{C} \PY{o}{=} \PY{n}{C}\PY{p}{,} \PY{n}{kernel} \PY{o}{=} \PY{l+s+s1}{\PYZsq{}}\PY{l+s+s1}{rbf}\PY{l+s+s1}{\PYZsq{}}\PY{p}{,} \PY{n}{gamma} \PY{o}{=} \PY{n}{gamma}\PY{p}{)}
                  \PY{n}{aux\PYZus{}kernel}\PY{o}{.}\PY{n}{fit}\PY{p}{(}\PY{n}{X\PYZus{}3}\PY{p}{,} \PY{n}{Y\PYZus{}3}\PY{o}{.}\PY{n}{flatten}\PY{p}{(}\PY{p}{)}\PY{p}{)}
                  \PY{n}{score} \PY{o}{=} \PY{n}{aux\PYZus{}kernel}\PY{o}{.}\PY{n}{score}\PY{p}{(}\PY{n}{X\PYZus{}Test}\PY{p}{,} \PY{n}{Y\PYZus{}Test}\PY{p}{)}
                  \PY{k}{if}\PY{p}{(}\PY{n}{score} \PY{o}{\PYZgt{}} \PY{n}{bestScore}\PY{p}{)}\PY{p}{:}
                      \PY{n}{bestC} \PY{o}{=} \PY{n}{C}
                      \PY{n}{bestSigma} \PY{o}{=} \PY{n}{sigma}
                      \PY{n}{bestScore} \PY{o}{=} \PY{n}{score}
                  
          \PY{n+nb}{print}\PY{p}{(}\PY{l+s+s2}{\PYZdq{}}\PY{l+s+s2}{Best score: }\PY{l+s+s2}{\PYZdq{}}\PY{p}{,} \PY{n}{bestScore}\PY{p}{)}
          \PY{n+nb}{print}\PY{p}{(}\PY{l+s+s2}{\PYZdq{}}\PY{l+s+s2}{Best sigma: }\PY{l+s+s2}{\PYZdq{}}\PY{p}{,} \PY{n}{bestSigma}\PY{p}{)}
          \PY{n+nb}{print}\PY{p}{(}\PY{l+s+s2}{\PYZdq{}}\PY{l+s+s2}{Best C:}\PY{l+s+s2}{\PYZdq{}}\PY{p}{,} \PY{n}{bestC}\PY{p}{)}
\end{Verbatim}


    \begin{Verbatim}[commandchars=\\\{\}]
Best score:  0.965
Best sigma:  0.1
Best C: 1.0

    \end{Verbatim}

    \begin{Verbatim}[commandchars=\\\{\}]
{\color{incolor}In [{\color{incolor}242}]:} \PY{n}{gKernel} \PY{o}{=} \PY{n}{svm}\PY{o}{.}\PY{n}{SVC} \PY{p}{(}\PY{n}{C} \PY{o}{=} \PY{n}{bestC}\PY{p}{,} \PY{n}{kernel} \PY{o}{=} \PY{l+s+s1}{\PYZsq{}}\PY{l+s+s1}{rbf}\PY{l+s+s1}{\PYZsq{}}\PY{p}{,} \PY{n}{gamma} \PY{o}{=}  \PY{l+m+mi}{1}\PY{o}{/}\PY{p}{(}\PY{l+m+mi}{2}\PY{o}{*}\PY{n}{bestSigma} \PY{o}{*}\PY{o}{*}\PY{l+m+mi}{2}\PY{p}{)}\PY{p}{)}
          \PY{n}{gKernel}\PY{o}{.}\PY{n}{fit}\PY{p}{(}\PY{n}{X\PYZus{}3}\PY{p}{,} \PY{n}{Y\PYZus{}3}\PY{o}{.}\PY{n}{flatten}\PY{p}{(}\PY{p}{)}\PY{p}{)}
          
          \PY{n}{plt}\PY{o}{.}\PY{n}{figure}\PY{p}{(}\PY{p}{)}
          \PY{n}{dibujaLimite}\PY{p}{(}\PY{n}{gKernel}\PY{p}{,} \PY{n}{pos\PYZus{}3}\PY{p}{,} \PY{n}{neg\PYZus{}3}\PY{p}{,} \PY{o}{\PYZhy{}}\PY{l+m+mf}{0.65}\PY{p}{,} \PY{l+m+mf}{0.3}\PY{p}{,} \PY{o}{\PYZhy{}}\PY{l+m+mf}{0.69}\PY{p}{,} \PY{l+m+mf}{0.61}\PY{p}{)}
          \PY{n}{plt}\PY{o}{.}\PY{n}{title}\PY{p}{(}\PY{l+s+s2}{\PYZdq{}}\PY{l+s+s2}{Mejor separación con kernel gaussiano}\PY{l+s+s2}{\PYZdq{}}\PY{p}{)}
          \PY{n}{plt}\PY{o}{.}\PY{n}{show}\PY{p}{(}\PY{p}{)}
\end{Verbatim}


    
    \begin{verbatim}
<Figure size 432x288 with 0 Axes>
    \end{verbatim}

    
    \begin{center}
    \adjustimage{max size={0.9\linewidth}{0.9\paperheight}}{output_26_1.png}
    \end{center}
    { \hspace*{\fill} \\}
    
    \subsection{Parte 4: Filtrado de Spam de
Correo}\label{parte-4-filtrado-de-spam-de-correo}

    En esta sección entrenaremos un SVM para ayudarnos a filtrar correo spam
de correo no spam.

    \subsubsection{Procesado del email}\label{procesado-del-email}

    Cargamos un correo y lo procesamos como una lista de tokens, para poder
entrenar el SVM.

    \begin{Verbatim}[commandchars=\\\{\}]
{\color{incolor}In [{\color{incolor}258}]:} \PY{k+kn}{import} \PY{n+nn}{process\PYZus{}email} \PY{k}{as} \PY{n+nn}{pmail}
          \PY{k+kn}{import} \PY{n+nn}{get\PYZus{}vocab\PYZus{}dict} \PY{k}{as} \PY{n+nn}{getvoc}
          \PY{k+kn}{import} \PY{n+nn}{codecs}
          
          \PY{n}{email\PYZus{}contents} \PY{o}{=} \PY{n}{codecs}\PY{o}{.}\PY{n}{open} \PY{p}{(}\PY{l+s+s2}{\PYZdq{}}\PY{l+s+s2}{spam/0001.txt}\PY{l+s+s2}{\PYZdq{}}\PY{p}{,} \PY{l+s+s1}{\PYZsq{}}\PY{l+s+s1}{r}\PY{l+s+s1}{\PYZsq{}}\PY{p}{)}\PY{o}{.}\PY{n}{read}\PY{p}{(}\PY{p}{)}
          \PY{n}{email} \PY{o}{=} \PY{n}{pmail}\PY{o}{.}\PY{n}{email2TokenList}\PY{p}{(}\PY{n}{email\PYZus{}contents}\PY{p}{)}
          \PY{n+nb}{print}\PY{p}{(}\PY{l+s+s2}{\PYZdq{}}\PY{l+s+s2}{Los correos se ven como listas de esta forma:}\PY{l+s+se}{\PYZbs{}n}\PY{l+s+s2}{\PYZdq{}}\PY{p}{,} \PY{n}{email}\PY{p}{)}
\end{Verbatim}


    \begin{Verbatim}[commandchars=\\\{\}]
Los correos se ven como listas de esta forma:
 ['save', 'up', 'to', 'number', 'on', 'life', 'insur', 'whi', 'spend', 'more', 'than', 'you', 'have', 'to', 'life', 'quot', 'save', 'ensurin', 'g', 'your', 'famili', 's', 'financi', 'secur', 'is', 'veri', 'import', 'life', 'quot', 'save', 'ma', 'ke', 'buy', 'life', 'insur', 'simpl', 'and', 'afford', 'we', 'provid', 'free', 'access', 'to', 'the', 'veri', 'best', 'compani', 'and', 'the', 'lowest', 'rate', 'life', 'quot', 'save', 'is', 'fast', 'ea', 'y', 'and', 'save', 'you', 'money', 'let', 'us', 'help', 'you', 'get', 'start', 'with', 'the', 'best', 'val', 'ue', 'in', 'the', 'countri', 'on', 'new', 'coverag', 'you', 'can', 'save', 'hundr', 'or', 'even', 'tho', 'usand', 'of', 'dollar', 'by', 'request', 'a', 'free', 'quot', 'from', 'lifequot', 'save', 'our', 'servic', 'will', 'take', 'you', 'less', 'than', 'number', 'minut', 'to', 'complet', 'shop', 'an', 'd', 'compar', 'save', 'up', 'to', 'number', 'on', 'all', 'type', 'of', 'life', 'insur', 'click', 'here', 'for', 'your', 'free', 'quot', 'protect', 'your', 'famili', 'is', 'the', 'best', 'invest', 'you', 'll', 'eve', 'r', 'make', 'if', 'you', 'are', 'in', 'receipt', 'of', 'thi', 'email', 'in', 'error', 'and', 'or', 'wish', 'to', 'be', 'remov', 'from', 'our', 'list', 'pleas', 'click', 'here', 'and', 'type', 'remov', 'if', 'you', 'resid', 'in', 'ani', 'state', 'which', 'prohibit', 'e', 'mail', 'solicit', 'for', 'insuran', 'ce', 'pleas', 'disregard', 'thi', 'email']

    \end{Verbatim}

    \begin{Verbatim}[commandchars=\\\{\}]
{\color{incolor}In [{\color{incolor}292}]:} \PY{k+kn}{import} \PY{n+nn}{collections}
          \PY{n}{vocabDict} \PY{o}{=} \PY{n}{collections}\PY{o}{.}\PY{n}{OrderedDict}\PY{p}{(}\PY{n}{getvoc}\PY{o}{.}\PY{n}{getVocabDict}\PY{p}{(}\PY{p}{)}\PY{p}{)}
          \PY{n}{X\PYZus{}input} \PY{o}{=} \PY{n}{np}\PY{o}{.}\PY{n}{zeros}\PY{p}{(}\PY{n}{shape}\PY{o}{=}\PY{p}{(}\PY{l+m+mi}{1000}\PY{p}{,} \PY{n+nb}{len}\PY{p}{(}\PY{n}{vocabDict}\PY{p}{)}\PY{p}{)}\PY{p}{)}
          \PY{n}{Y} \PY{o}{=} \PY{n}{np}\PY{o}{.}\PY{n}{zeros}\PY{p}{(}\PY{l+m+mi}{1000}\PY{p}{)}
          
          \PY{c+c1}{\PYZsh{} Los 500 primeros valores son spam}
          \PY{k}{for} \PY{n}{i} \PY{o+ow}{in} \PY{n+nb}{range}\PY{p}{(}\PY{l+m+mi}{1}\PY{p}{,}\PY{l+m+mi}{500}\PY{p}{)}\PY{p}{:}
              \PY{n}{Y}\PY{p}{[}\PY{n}{i}\PY{p}{]} \PY{o}{=} \PY{l+m+mi}{1}
              \PY{n}{path} \PY{o}{=} \PY{l+s+s2}{\PYZdq{}}\PY{l+s+s2}{spam/}\PY{l+s+si}{\PYZob{}:04d\PYZcb{}}\PY{l+s+s2}{.txt}\PY{l+s+s2}{\PYZdq{}}\PY{o}{.}\PY{n}{format}\PY{p}{(}\PY{n}{i}\PY{p}{)}
              \PY{n}{email} \PY{o}{=} \PY{n}{pmail}\PY{o}{.}\PY{n}{email2TokenList}\PY{p}{(}\PY{n}{codecs}\PY{o}{.}\PY{n}{open} \PY{p}{(}\PY{n}{path}\PY{p}{,} \PY{l+s+s1}{\PYZsq{}}\PY{l+s+s1}{r}\PY{l+s+s1}{\PYZsq{}}\PY{p}{,} \PY{n}{encoding} \PY{o}{=} \PY{l+s+s1}{\PYZsq{}}\PY{l+s+s1}{utf\PYZhy{}8}\PY{l+s+s1}{\PYZsq{}}\PY{p}{,} \PY{n}{errors} \PY{o}{=} \PY{l+s+s1}{\PYZsq{}}\PY{l+s+s1}{ignore}\PY{l+s+s1}{\PYZsq{}}\PY{p}{)}\PY{o}{.}\PY{n}{read}\PY{p}{(}\PY{p}{)}\PY{p}{)}
              \PY{k}{for} \PY{n}{j} \PY{o+ow}{in} \PY{n+nb}{range} \PY{p}{(}\PY{n+nb}{len}\PY{p}{(}\PY{n}{email}\PY{p}{)}\PY{p}{)}\PY{p}{:}
                  \PY{k}{if}\PY{p}{(}\PY{n}{email} \PY{p}{[}\PY{n}{j}\PY{p}{]} \PY{o+ow}{in} \PY{n}{vocabDict}\PY{p}{)}\PY{p}{:}
                      \PY{n}{X\PYZus{}input}\PY{p}{[}\PY{n}{i}\PY{p}{]}\PY{p}{[}\PY{n+nb}{list}\PY{p}{(}\PY{n}{vocabDict}\PY{o}{.}\PY{n}{keys}\PY{p}{(}\PY{p}{)}\PY{p}{)}\PY{o}{.}\PY{n}{index}\PY{p}{(}\PY{n}{email}\PY{p}{[}\PY{n}{j}\PY{p}{]}\PY{p}{)}\PY{p}{]} \PY{o}{=} \PY{l+m+mi}{1}
                      
                      \PY{c+c1}{\PYZsh{} los 500 restantes no lo son}
          \PY{k}{for} \PY{n}{i} \PY{o+ow}{in} \PY{n+nb}{range}\PY{p}{(}\PY{l+m+mi}{1}\PY{p}{,}\PY{l+m+mi}{500}\PY{p}{)}\PY{p}{:}
              \PY{n}{Y}\PY{p}{[}\PY{l+m+mi}{500}\PY{o}{+}\PY{n}{i}\PY{p}{]} \PY{o}{=} \PY{l+m+mi}{0}
              \PY{n}{path} \PY{o}{=} \PY{l+s+s2}{\PYZdq{}}\PY{l+s+s2}{spam/}\PY{l+s+si}{\PYZob{}:04d\PYZcb{}}\PY{l+s+s2}{.txt}\PY{l+s+s2}{\PYZdq{}}\PY{o}{.}\PY{n}{format}\PY{p}{(}\PY{n}{i}\PY{p}{)}
              \PY{n}{email} \PY{o}{=} \PY{n}{pmail}\PY{o}{.}\PY{n}{email2TokenList}\PY{p}{(}\PY{n}{codecs}\PY{o}{.}\PY{n}{open} \PY{p}{(}\PY{n}{path}\PY{p}{,} \PY{l+s+s1}{\PYZsq{}}\PY{l+s+s1}{r}\PY{l+s+s1}{\PYZsq{}}\PY{p}{,} \PY{n}{encoding} \PY{o}{=} \PY{l+s+s1}{\PYZsq{}}\PY{l+s+s1}{utf\PYZhy{}8}\PY{l+s+s1}{\PYZsq{}}\PY{p}{,} \PY{n}{errors} \PY{o}{=} \PY{l+s+s1}{\PYZsq{}}\PY{l+s+s1}{ignore}\PY{l+s+s1}{\PYZsq{}}\PY{p}{)}\PY{o}{.}\PY{n}{read}\PY{p}{(}\PY{p}{)}\PY{p}{)}
              \PY{k}{for} \PY{n}{j} \PY{o+ow}{in} \PY{n+nb}{range} \PY{p}{(}\PY{n+nb}{len}\PY{p}{(}\PY{n}{email}\PY{p}{)}\PY{p}{)}\PY{p}{:}
                  \PY{k}{if}\PY{p}{(}\PY{n}{email} \PY{p}{[}\PY{n}{j}\PY{p}{]} \PY{o+ow}{in} \PY{n}{vocabDict}\PY{p}{)}\PY{p}{:}
                      \PY{n}{X\PYZus{}input}\PY{p}{[}\PY{l+m+mi}{500}\PY{o}{+}\PY{n}{i}\PY{p}{]}\PY{p}{[}\PY{n+nb}{list}\PY{p}{(}\PY{n}{vocabDict}\PY{o}{.}\PY{n}{keys}\PY{p}{(}\PY{p}{)}\PY{p}{)}\PY{o}{.}\PY{n}{index}\PY{p}{(}\PY{n}{email}\PY{p}{[}\PY{n}{j}\PY{p}{]}\PY{p}{)}\PY{p}{]} \PY{o}{=} \PY{l+m+mi}{1}
          
                       
                  
\end{Verbatim}


    \begin{Verbatim}[commandchars=\\\{\}]
{\color{incolor}In [{\color{incolor}293}]:} \PY{n}{spam\PYZus{}processor} \PY{o}{=} \PY{n}{svm}\PY{o}{.}\PY{n}{SVC}\PY{p}{(}\PY{n}{C} \PY{o}{=} \PY{l+m+mf}{1.0}\PY{p}{,} \PY{n}{kernel} \PY{o}{=} \PY{l+s+s1}{\PYZsq{}}\PY{l+s+s1}{linear}\PY{l+s+s1}{\PYZsq{}}\PY{p}{)}
          \PY{n}{spam\PYZus{}processor}\PY{o}{.}\PY{n}{fit}\PY{p}{(}\PY{n}{X\PYZus{}input}\PY{p}{,} \PY{n}{Y}\PY{o}{.}\PY{n}{ravel}\PY{p}{(}\PY{p}{)}\PY{p}{)}
\end{Verbatim}


\begin{Verbatim}[commandchars=\\\{\}]
{\color{outcolor}Out[{\color{outcolor}293}]:} SVC(C=1.0, cache\_size=200, class\_weight=None, coef0=0.0,
            decision\_function\_shape='ovr', degree=3, gamma='auto', kernel='linear',
            max\_iter=-1, probability=False, random\_state=None, shrinking=True,
            tol=0.001, verbose=False)
\end{Verbatim}
            
    \begin{Verbatim}[commandchars=\\\{\}]
{\color{incolor}In [{\color{incolor}294}]:} 
\end{Verbatim}


    \begin{Verbatim}[commandchars=\\\{\}]

        ---------------------------------------------------------------------------

        ValueError                                Traceback (most recent call last)

        <ipython-input-294-44e92776962e> in <module>()
    ----> 1 pred = spam\_processor.predict(X\_input[0])
          2 print(pred)
    

        C:\textbackslash{}Program Files\textbackslash{}Anaconda\textbackslash{}lib\textbackslash{}site-packages\textbackslash{}sklearn\textbackslash{}svm\textbackslash{}base.py in predict(self, X)
        546             Class labels for samples in X.
        547         """
    --> 548         y = super(BaseSVC, self).predict(X)
        549         return self.classes\_.take(np.asarray(y, dtype=np.intp))
        550 
    

        C:\textbackslash{}Program Files\textbackslash{}Anaconda\textbackslash{}lib\textbackslash{}site-packages\textbackslash{}sklearn\textbackslash{}svm\textbackslash{}base.py in predict(self, X)
        306         y\_pred : array, shape (n\_samples,)
        307         """
    --> 308         X = self.\_validate\_for\_predict(X)
        309         predict = self.\_sparse\_predict if self.\_sparse else self.\_dense\_predict
        310         return predict(X)
    

        C:\textbackslash{}Program Files\textbackslash{}Anaconda\textbackslash{}lib\textbackslash{}site-packages\textbackslash{}sklearn\textbackslash{}svm\textbackslash{}base.py in \_validate\_for\_predict(self, X)
        437         check\_is\_fitted(self, 'support\_')
        438 
    --> 439         X = check\_array(X, accept\_sparse='csr', dtype=np.float64, order="C")
        440         if self.\_sparse and not sp.isspmatrix(X):
        441             X = sp.csr\_matrix(X)
    

        C:\textbackslash{}Program Files\textbackslash{}Anaconda\textbackslash{}lib\textbackslash{}site-packages\textbackslash{}sklearn\textbackslash{}utils\textbackslash{}validation.py in check\_array(array, accept\_sparse, dtype, order, copy, force\_all\_finite, ensure\_2d, allow\_nd, ensure\_min\_samples, ensure\_min\_features, warn\_on\_dtype, estimator)
        439                     "Reshape your data either using array.reshape(-1, 1) if "
        440                     "your data has a single feature or array.reshape(1, -1) "
    --> 441                     "if it contains a single sample.".format(array))
        442             array = np.atleast\_2d(array)
        443             \# To ensure that array flags are maintained
    

        ValueError: Expected 2D array, got 1D array instead:
    array=[0. 0. 0. {\ldots} 0. 0. 0.].
    Reshape your data either using array.reshape(-1, 1) if your data has a single feature or array.reshape(1, -1) if it contains a single sample.

    \end{Verbatim}


    % Add a bibliography block to the postdoc
    
    
    
    \end{document}
