
% Default to the notebook output style

    


% Inherit from the specified cell style.




    
\documentclass[11pt]{article}

    
    
    \usepackage[T1]{fontenc}
    % Nicer default font (+ math font) than Computer Modern for most use cases
    \usepackage{mathpazo}

    % Basic figure setup, for now with no caption control since it's done
    % automatically by Pandoc (which extracts ![](path) syntax from Markdown).
    \usepackage{graphicx}
    % We will generate all images so they have a width \maxwidth. This means
    % that they will get their normal width if they fit onto the page, but
    % are scaled down if they would overflow the margins.
    \makeatletter
    \def\maxwidth{\ifdim\Gin@nat@width>\linewidth\linewidth
    \else\Gin@nat@width\fi}
    \makeatother
    \let\Oldincludegraphics\includegraphics
    % Set max figure width to be 80% of text width, for now hardcoded.
    \renewcommand{\includegraphics}[1]{\Oldincludegraphics[width=.8\maxwidth]{#1}}
    % Ensure that by default, figures have no caption (until we provide a
    % proper Figure object with a Caption API and a way to capture that
    % in the conversion process - todo).
    \usepackage{caption}
    \DeclareCaptionLabelFormat{nolabel}{}
    \captionsetup{labelformat=nolabel}

    \usepackage{adjustbox} % Used to constrain images to a maximum size 
    \usepackage{xcolor} % Allow colors to be defined
    \usepackage{enumerate} % Needed for markdown enumerations to work
    \usepackage{geometry} % Used to adjust the document margins
    \usepackage{amsmath} % Equations
    \usepackage{amssymb} % Equations
    \usepackage{textcomp} % defines textquotesingle
    % Hack from http://tex.stackexchange.com/a/47451/13684:
    \AtBeginDocument{%
        \def\PYZsq{\textquotesingle}% Upright quotes in Pygmentized code
    }
    \usepackage{upquote} % Upright quotes for verbatim code
    \usepackage{eurosym} % defines \euro
    \usepackage[mathletters]{ucs} % Extended unicode (utf-8) support
    \usepackage[utf8x]{inputenc} % Allow utf-8 characters in the tex document
    \usepackage{fancyvrb} % verbatim replacement that allows latex
    \usepackage{grffile} % extends the file name processing of package graphics 
                         % to support a larger range 
    % The hyperref package gives us a pdf with properly built
    % internal navigation ('pdf bookmarks' for the table of contents,
    % internal cross-reference links, web links for URLs, etc.)
    \usepackage{hyperref}
    \usepackage{longtable} % longtable support required by pandoc >1.10
    \usepackage{booktabs}  % table support for pandoc > 1.12.2
    \usepackage[inline]{enumitem} % IRkernel/repr support (it uses the enumerate* environment)
    \usepackage[normalem]{ulem} % ulem is needed to support strikethroughs (\sout)
                                % normalem makes italics be italics, not underlines
    

    
    
    % Colors for the hyperref package
    \definecolor{urlcolor}{rgb}{0,.145,.698}
    \definecolor{linkcolor}{rgb}{.71,0.21,0.01}
    \definecolor{citecolor}{rgb}{.12,.54,.11}

    % ANSI colors
    \definecolor{ansi-black}{HTML}{3E424D}
    \definecolor{ansi-black-intense}{HTML}{282C36}
    \definecolor{ansi-red}{HTML}{E75C58}
    \definecolor{ansi-red-intense}{HTML}{B22B31}
    \definecolor{ansi-green}{HTML}{00A250}
    \definecolor{ansi-green-intense}{HTML}{007427}
    \definecolor{ansi-yellow}{HTML}{DDB62B}
    \definecolor{ansi-yellow-intense}{HTML}{B27D12}
    \definecolor{ansi-blue}{HTML}{208FFB}
    \definecolor{ansi-blue-intense}{HTML}{0065CA}
    \definecolor{ansi-magenta}{HTML}{D160C4}
    \definecolor{ansi-magenta-intense}{HTML}{A03196}
    \definecolor{ansi-cyan}{HTML}{60C6C8}
    \definecolor{ansi-cyan-intense}{HTML}{258F8F}
    \definecolor{ansi-white}{HTML}{C5C1B4}
    \definecolor{ansi-white-intense}{HTML}{A1A6B2}

    % commands and environments needed by pandoc snippets
    % extracted from the output of `pandoc -s`
    \providecommand{\tightlist}{%
      \setlength{\itemsep}{0pt}\setlength{\parskip}{0pt}}
    \DefineVerbatimEnvironment{Highlighting}{Verbatim}{commandchars=\\\{\}}
    % Add ',fontsize=\small' for more characters per line
    \newenvironment{Shaded}{}{}
    \newcommand{\KeywordTok}[1]{\textcolor[rgb]{0.00,0.44,0.13}{\textbf{{#1}}}}
    \newcommand{\DataTypeTok}[1]{\textcolor[rgb]{0.56,0.13,0.00}{{#1}}}
    \newcommand{\DecValTok}[1]{\textcolor[rgb]{0.25,0.63,0.44}{{#1}}}
    \newcommand{\BaseNTok}[1]{\textcolor[rgb]{0.25,0.63,0.44}{{#1}}}
    \newcommand{\FloatTok}[1]{\textcolor[rgb]{0.25,0.63,0.44}{{#1}}}
    \newcommand{\CharTok}[1]{\textcolor[rgb]{0.25,0.44,0.63}{{#1}}}
    \newcommand{\StringTok}[1]{\textcolor[rgb]{0.25,0.44,0.63}{{#1}}}
    \newcommand{\CommentTok}[1]{\textcolor[rgb]{0.38,0.63,0.69}{\textit{{#1}}}}
    \newcommand{\OtherTok}[1]{\textcolor[rgb]{0.00,0.44,0.13}{{#1}}}
    \newcommand{\AlertTok}[1]{\textcolor[rgb]{1.00,0.00,0.00}{\textbf{{#1}}}}
    \newcommand{\FunctionTok}[1]{\textcolor[rgb]{0.02,0.16,0.49}{{#1}}}
    \newcommand{\RegionMarkerTok}[1]{{#1}}
    \newcommand{\ErrorTok}[1]{\textcolor[rgb]{1.00,0.00,0.00}{\textbf{{#1}}}}
    \newcommand{\NormalTok}[1]{{#1}}
    
    % Additional commands for more recent versions of Pandoc
    \newcommand{\ConstantTok}[1]{\textcolor[rgb]{0.53,0.00,0.00}{{#1}}}
    \newcommand{\SpecialCharTok}[1]{\textcolor[rgb]{0.25,0.44,0.63}{{#1}}}
    \newcommand{\VerbatimStringTok}[1]{\textcolor[rgb]{0.25,0.44,0.63}{{#1}}}
    \newcommand{\SpecialStringTok}[1]{\textcolor[rgb]{0.73,0.40,0.53}{{#1}}}
    \newcommand{\ImportTok}[1]{{#1}}
    \newcommand{\DocumentationTok}[1]{\textcolor[rgb]{0.73,0.13,0.13}{\textit{{#1}}}}
    \newcommand{\AnnotationTok}[1]{\textcolor[rgb]{0.38,0.63,0.69}{\textbf{\textit{{#1}}}}}
    \newcommand{\CommentVarTok}[1]{\textcolor[rgb]{0.38,0.63,0.69}{\textbf{\textit{{#1}}}}}
    \newcommand{\VariableTok}[1]{\textcolor[rgb]{0.10,0.09,0.49}{{#1}}}
    \newcommand{\ControlFlowTok}[1]{\textcolor[rgb]{0.00,0.44,0.13}{\textbf{{#1}}}}
    \newcommand{\OperatorTok}[1]{\textcolor[rgb]{0.40,0.40,0.40}{{#1}}}
    \newcommand{\BuiltInTok}[1]{{#1}}
    \newcommand{\ExtensionTok}[1]{{#1}}
    \newcommand{\PreprocessorTok}[1]{\textcolor[rgb]{0.74,0.48,0.00}{{#1}}}
    \newcommand{\AttributeTok}[1]{\textcolor[rgb]{0.49,0.56,0.16}{{#1}}}
    \newcommand{\InformationTok}[1]{\textcolor[rgb]{0.38,0.63,0.69}{\textbf{\textit{{#1}}}}}
    \newcommand{\WarningTok}[1]{\textcolor[rgb]{0.38,0.63,0.69}{\textbf{\textit{{#1}}}}}
    
    
    % Define a nice break command that doesn't care if a line doesn't already
    % exist.
    \def\br{\hspace*{\fill} \\* }
    % Math Jax compatability definitions
    \def\gt{>}
    \def\lt{<}
    % Document parameters
    \title{P4}
    
    
    

    % Pygments definitions
    
\makeatletter
\def\PY@reset{\let\PY@it=\relax \let\PY@bf=\relax%
    \let\PY@ul=\relax \let\PY@tc=\relax%
    \let\PY@bc=\relax \let\PY@ff=\relax}
\def\PY@tok#1{\csname PY@tok@#1\endcsname}
\def\PY@toks#1+{\ifx\relax#1\empty\else%
    \PY@tok{#1}\expandafter\PY@toks\fi}
\def\PY@do#1{\PY@bc{\PY@tc{\PY@ul{%
    \PY@it{\PY@bf{\PY@ff{#1}}}}}}}
\def\PY#1#2{\PY@reset\PY@toks#1+\relax+\PY@do{#2}}

\expandafter\def\csname PY@tok@w\endcsname{\def\PY@tc##1{\textcolor[rgb]{0.73,0.73,0.73}{##1}}}
\expandafter\def\csname PY@tok@c\endcsname{\let\PY@it=\textit\def\PY@tc##1{\textcolor[rgb]{0.25,0.50,0.50}{##1}}}
\expandafter\def\csname PY@tok@cp\endcsname{\def\PY@tc##1{\textcolor[rgb]{0.74,0.48,0.00}{##1}}}
\expandafter\def\csname PY@tok@k\endcsname{\let\PY@bf=\textbf\def\PY@tc##1{\textcolor[rgb]{0.00,0.50,0.00}{##1}}}
\expandafter\def\csname PY@tok@kp\endcsname{\def\PY@tc##1{\textcolor[rgb]{0.00,0.50,0.00}{##1}}}
\expandafter\def\csname PY@tok@kt\endcsname{\def\PY@tc##1{\textcolor[rgb]{0.69,0.00,0.25}{##1}}}
\expandafter\def\csname PY@tok@o\endcsname{\def\PY@tc##1{\textcolor[rgb]{0.40,0.40,0.40}{##1}}}
\expandafter\def\csname PY@tok@ow\endcsname{\let\PY@bf=\textbf\def\PY@tc##1{\textcolor[rgb]{0.67,0.13,1.00}{##1}}}
\expandafter\def\csname PY@tok@nb\endcsname{\def\PY@tc##1{\textcolor[rgb]{0.00,0.50,0.00}{##1}}}
\expandafter\def\csname PY@tok@nf\endcsname{\def\PY@tc##1{\textcolor[rgb]{0.00,0.00,1.00}{##1}}}
\expandafter\def\csname PY@tok@nc\endcsname{\let\PY@bf=\textbf\def\PY@tc##1{\textcolor[rgb]{0.00,0.00,1.00}{##1}}}
\expandafter\def\csname PY@tok@nn\endcsname{\let\PY@bf=\textbf\def\PY@tc##1{\textcolor[rgb]{0.00,0.00,1.00}{##1}}}
\expandafter\def\csname PY@tok@ne\endcsname{\let\PY@bf=\textbf\def\PY@tc##1{\textcolor[rgb]{0.82,0.25,0.23}{##1}}}
\expandafter\def\csname PY@tok@nv\endcsname{\def\PY@tc##1{\textcolor[rgb]{0.10,0.09,0.49}{##1}}}
\expandafter\def\csname PY@tok@no\endcsname{\def\PY@tc##1{\textcolor[rgb]{0.53,0.00,0.00}{##1}}}
\expandafter\def\csname PY@tok@nl\endcsname{\def\PY@tc##1{\textcolor[rgb]{0.63,0.63,0.00}{##1}}}
\expandafter\def\csname PY@tok@ni\endcsname{\let\PY@bf=\textbf\def\PY@tc##1{\textcolor[rgb]{0.60,0.60,0.60}{##1}}}
\expandafter\def\csname PY@tok@na\endcsname{\def\PY@tc##1{\textcolor[rgb]{0.49,0.56,0.16}{##1}}}
\expandafter\def\csname PY@tok@nt\endcsname{\let\PY@bf=\textbf\def\PY@tc##1{\textcolor[rgb]{0.00,0.50,0.00}{##1}}}
\expandafter\def\csname PY@tok@nd\endcsname{\def\PY@tc##1{\textcolor[rgb]{0.67,0.13,1.00}{##1}}}
\expandafter\def\csname PY@tok@s\endcsname{\def\PY@tc##1{\textcolor[rgb]{0.73,0.13,0.13}{##1}}}
\expandafter\def\csname PY@tok@sd\endcsname{\let\PY@it=\textit\def\PY@tc##1{\textcolor[rgb]{0.73,0.13,0.13}{##1}}}
\expandafter\def\csname PY@tok@si\endcsname{\let\PY@bf=\textbf\def\PY@tc##1{\textcolor[rgb]{0.73,0.40,0.53}{##1}}}
\expandafter\def\csname PY@tok@se\endcsname{\let\PY@bf=\textbf\def\PY@tc##1{\textcolor[rgb]{0.73,0.40,0.13}{##1}}}
\expandafter\def\csname PY@tok@sr\endcsname{\def\PY@tc##1{\textcolor[rgb]{0.73,0.40,0.53}{##1}}}
\expandafter\def\csname PY@tok@ss\endcsname{\def\PY@tc##1{\textcolor[rgb]{0.10,0.09,0.49}{##1}}}
\expandafter\def\csname PY@tok@sx\endcsname{\def\PY@tc##1{\textcolor[rgb]{0.00,0.50,0.00}{##1}}}
\expandafter\def\csname PY@tok@m\endcsname{\def\PY@tc##1{\textcolor[rgb]{0.40,0.40,0.40}{##1}}}
\expandafter\def\csname PY@tok@gh\endcsname{\let\PY@bf=\textbf\def\PY@tc##1{\textcolor[rgb]{0.00,0.00,0.50}{##1}}}
\expandafter\def\csname PY@tok@gu\endcsname{\let\PY@bf=\textbf\def\PY@tc##1{\textcolor[rgb]{0.50,0.00,0.50}{##1}}}
\expandafter\def\csname PY@tok@gd\endcsname{\def\PY@tc##1{\textcolor[rgb]{0.63,0.00,0.00}{##1}}}
\expandafter\def\csname PY@tok@gi\endcsname{\def\PY@tc##1{\textcolor[rgb]{0.00,0.63,0.00}{##1}}}
\expandafter\def\csname PY@tok@gr\endcsname{\def\PY@tc##1{\textcolor[rgb]{1.00,0.00,0.00}{##1}}}
\expandafter\def\csname PY@tok@ge\endcsname{\let\PY@it=\textit}
\expandafter\def\csname PY@tok@gs\endcsname{\let\PY@bf=\textbf}
\expandafter\def\csname PY@tok@gp\endcsname{\let\PY@bf=\textbf\def\PY@tc##1{\textcolor[rgb]{0.00,0.00,0.50}{##1}}}
\expandafter\def\csname PY@tok@go\endcsname{\def\PY@tc##1{\textcolor[rgb]{0.53,0.53,0.53}{##1}}}
\expandafter\def\csname PY@tok@gt\endcsname{\def\PY@tc##1{\textcolor[rgb]{0.00,0.27,0.87}{##1}}}
\expandafter\def\csname PY@tok@err\endcsname{\def\PY@bc##1{\setlength{\fboxsep}{0pt}\fcolorbox[rgb]{1.00,0.00,0.00}{1,1,1}{\strut ##1}}}
\expandafter\def\csname PY@tok@kc\endcsname{\let\PY@bf=\textbf\def\PY@tc##1{\textcolor[rgb]{0.00,0.50,0.00}{##1}}}
\expandafter\def\csname PY@tok@kd\endcsname{\let\PY@bf=\textbf\def\PY@tc##1{\textcolor[rgb]{0.00,0.50,0.00}{##1}}}
\expandafter\def\csname PY@tok@kn\endcsname{\let\PY@bf=\textbf\def\PY@tc##1{\textcolor[rgb]{0.00,0.50,0.00}{##1}}}
\expandafter\def\csname PY@tok@kr\endcsname{\let\PY@bf=\textbf\def\PY@tc##1{\textcolor[rgb]{0.00,0.50,0.00}{##1}}}
\expandafter\def\csname PY@tok@bp\endcsname{\def\PY@tc##1{\textcolor[rgb]{0.00,0.50,0.00}{##1}}}
\expandafter\def\csname PY@tok@fm\endcsname{\def\PY@tc##1{\textcolor[rgb]{0.00,0.00,1.00}{##1}}}
\expandafter\def\csname PY@tok@vc\endcsname{\def\PY@tc##1{\textcolor[rgb]{0.10,0.09,0.49}{##1}}}
\expandafter\def\csname PY@tok@vg\endcsname{\def\PY@tc##1{\textcolor[rgb]{0.10,0.09,0.49}{##1}}}
\expandafter\def\csname PY@tok@vi\endcsname{\def\PY@tc##1{\textcolor[rgb]{0.10,0.09,0.49}{##1}}}
\expandafter\def\csname PY@tok@vm\endcsname{\def\PY@tc##1{\textcolor[rgb]{0.10,0.09,0.49}{##1}}}
\expandafter\def\csname PY@tok@sa\endcsname{\def\PY@tc##1{\textcolor[rgb]{0.73,0.13,0.13}{##1}}}
\expandafter\def\csname PY@tok@sb\endcsname{\def\PY@tc##1{\textcolor[rgb]{0.73,0.13,0.13}{##1}}}
\expandafter\def\csname PY@tok@sc\endcsname{\def\PY@tc##1{\textcolor[rgb]{0.73,0.13,0.13}{##1}}}
\expandafter\def\csname PY@tok@dl\endcsname{\def\PY@tc##1{\textcolor[rgb]{0.73,0.13,0.13}{##1}}}
\expandafter\def\csname PY@tok@s2\endcsname{\def\PY@tc##1{\textcolor[rgb]{0.73,0.13,0.13}{##1}}}
\expandafter\def\csname PY@tok@sh\endcsname{\def\PY@tc##1{\textcolor[rgb]{0.73,0.13,0.13}{##1}}}
\expandafter\def\csname PY@tok@s1\endcsname{\def\PY@tc##1{\textcolor[rgb]{0.73,0.13,0.13}{##1}}}
\expandafter\def\csname PY@tok@mb\endcsname{\def\PY@tc##1{\textcolor[rgb]{0.40,0.40,0.40}{##1}}}
\expandafter\def\csname PY@tok@mf\endcsname{\def\PY@tc##1{\textcolor[rgb]{0.40,0.40,0.40}{##1}}}
\expandafter\def\csname PY@tok@mh\endcsname{\def\PY@tc##1{\textcolor[rgb]{0.40,0.40,0.40}{##1}}}
\expandafter\def\csname PY@tok@mi\endcsname{\def\PY@tc##1{\textcolor[rgb]{0.40,0.40,0.40}{##1}}}
\expandafter\def\csname PY@tok@il\endcsname{\def\PY@tc##1{\textcolor[rgb]{0.40,0.40,0.40}{##1}}}
\expandafter\def\csname PY@tok@mo\endcsname{\def\PY@tc##1{\textcolor[rgb]{0.40,0.40,0.40}{##1}}}
\expandafter\def\csname PY@tok@ch\endcsname{\let\PY@it=\textit\def\PY@tc##1{\textcolor[rgb]{0.25,0.50,0.50}{##1}}}
\expandafter\def\csname PY@tok@cm\endcsname{\let\PY@it=\textit\def\PY@tc##1{\textcolor[rgb]{0.25,0.50,0.50}{##1}}}
\expandafter\def\csname PY@tok@cpf\endcsname{\let\PY@it=\textit\def\PY@tc##1{\textcolor[rgb]{0.25,0.50,0.50}{##1}}}
\expandafter\def\csname PY@tok@c1\endcsname{\let\PY@it=\textit\def\PY@tc##1{\textcolor[rgb]{0.25,0.50,0.50}{##1}}}
\expandafter\def\csname PY@tok@cs\endcsname{\let\PY@it=\textit\def\PY@tc##1{\textcolor[rgb]{0.25,0.50,0.50}{##1}}}

\def\PYZbs{\char`\\}
\def\PYZus{\char`\_}
\def\PYZob{\char`\{}
\def\PYZcb{\char`\}}
\def\PYZca{\char`\^}
\def\PYZam{\char`\&}
\def\PYZlt{\char`\<}
\def\PYZgt{\char`\>}
\def\PYZsh{\char`\#}
\def\PYZpc{\char`\%}
\def\PYZdl{\char`\$}
\def\PYZhy{\char`\-}
\def\PYZsq{\char`\'}
\def\PYZdq{\char`\"}
\def\PYZti{\char`\~}
% for compatibility with earlier versions
\def\PYZat{@}
\def\PYZlb{[}
\def\PYZrb{]}
\makeatother


    % Exact colors from NB
    \definecolor{incolor}{rgb}{0.0, 0.0, 0.5}
    \definecolor{outcolor}{rgb}{0.545, 0.0, 0.0}



    
    % Prevent overflowing lines due to hard-to-break entities
    \sloppy 
    % Setup hyperref package
    \hypersetup{
      breaklinks=true,  % so long urls are correctly broken across lines
      colorlinks=true,
      urlcolor=urlcolor,
      linkcolor=linkcolor,
      citecolor=citecolor,
      }
    % Slightly bigger margins than the latex defaults
    
    \geometry{verbose,tmargin=1in,bmargin=1in,lmargin=1in,rmargin=1in}
    
    

    \begin{document}
    
    
    \maketitle
    
    

    
    \section{Practica 4 Aprendizaje Automático y Minería de
Datos}\label{practica-4-aprendizaje-automuxe1tico-y-mineruxeda-de-datos}

\subsubsection{Mario Jimenez y Manuel
Hernández}\label{mario-jimenez-y-manuel-hernuxe1ndez}

    Esta práctica consiste en reconocer digitos manuscritos mediante una red
neuronal.

    \subsection{Inclusión de librerías}\label{inclusiuxf3n-de-libreruxedas}

    \begin{Verbatim}[commandchars=\\\{\}]
{\color{incolor}In [{\color{incolor}35}]:} \PY{k+kn}{import} \PY{n+nn}{displayData}
         \PY{k+kn}{import} \PY{n+nn}{numpy} \PY{k}{as} \PY{n+nn}{np}
         \PY{k+kn}{import} \PY{n+nn}{displayData} \PY{k}{as} \PY{n+nn}{dp}
         \PY{k+kn}{from} \PY{n+nn}{scipy}\PY{n+nn}{.}\PY{n+nn}{io} \PY{k}{import} \PY{n}{loadmat}
         \PY{k+kn}{import} \PY{n+nn}{matplotlib}\PY{n+nn}{.}\PY{n+nn}{pyplot} \PY{k}{as} \PY{n+nn}{plt}
         \PY{k+kn}{import} \PY{n+nn}{scipy}\PY{n+nn}{.}\PY{n+nn}{optimize} \PY{k}{as} \PY{n+nn}{opt}
         \PY{k+kn}{import} \PY{n+nn}{checkNNGradients} \PY{k}{as} \PY{n+nn}{check}
\end{Verbatim}


    \subsection{Lectura de datos}\label{lectura-de-datos}

    \begin{Verbatim}[commandchars=\\\{\}]
{\color{incolor}In [{\color{incolor}36}]:} \PY{n}{data} \PY{o}{=} \PY{n}{loadmat}\PY{p}{(}\PY{l+s+s1}{\PYZsq{}}\PY{l+s+s1}{ex4data1.mat}\PY{l+s+s1}{\PYZsq{}}\PY{p}{)}
         \PY{n}{Y} \PY{o}{=} \PY{n}{data}\PY{p}{[}\PY{l+s+s1}{\PYZsq{}}\PY{l+s+s1}{y}\PY{l+s+s1}{\PYZsq{}}\PY{p}{]}  \PY{c+c1}{\PYZsh{} Representa el valor real de cada ejemplo de entrenamiento de X (y para cada X)}
         \PY{n}{X} \PY{o}{=} \PY{n}{data}\PY{p}{[}\PY{l+s+s1}{\PYZsq{}}\PY{l+s+s1}{X}\PY{l+s+s1}{\PYZsq{}}\PY{p}{]}  \PY{c+c1}{\PYZsh{} Cada fila de X representa una escala de grises de 20x20 desplegada linearmente (400 pixeles)}
         \PY{n}{nMuestras} \PY{o}{=} \PY{n+nb}{len}\PY{p}{(}\PY{n}{X}\PY{p}{)}
         \PY{n}{Y} \PY{o}{=} \PY{n}{np}\PY{o}{.}\PY{n}{ravel}\PY{p}{(}\PY{n}{Y}\PY{p}{)}
\end{Verbatim}


    Las entradas son mapas de bits de 20x20, que se desdoblan en columnas de
400 elementos a los que se les asocia un valor.

    \begin{Verbatim}[commandchars=\\\{\}]
{\color{incolor}In [{\color{incolor}37}]:} \PY{n+nb}{print}\PY{p}{(}\PY{n}{Y}\PY{p}{[}\PY{l+m+mi}{4999}\PY{p}{]}\PY{p}{)}
         \PY{n}{plt}\PY{o}{.}\PY{n}{figure}\PY{p}{(}\PY{p}{)}
         \PY{n}{dp}\PY{o}{.}\PY{n}{displayImage}\PY{p}{(}\PY{n}{X}\PY{p}{[}\PY{l+m+mi}{4999}\PY{p}{]}\PY{p}{)}
         \PY{n}{plt}\PY{o}{.}\PY{n}{savefig}\PY{p}{(}\PY{l+s+s2}{\PYZdq{}}\PY{l+s+s2}{Input\PYZus{}sample}\PY{l+s+s2}{\PYZdq{}}\PY{p}{)}
         \PY{n}{plt}\PY{o}{.}\PY{n}{show}\PY{p}{(}\PY{p}{)}
\end{Verbatim}


    \begin{Verbatim}[commandchars=\\\{\}]
9

    \end{Verbatim}

    
    \begin{verbatim}
<Figure size 432x288 with 0 Axes>
    \end{verbatim}

    
    \begin{center}
    \adjustimage{max size={0.9\linewidth}{0.9\paperheight}}{output_7_2.png}
    \end{center}
    { \hspace*{\fill} \\}
    
    \subsubsection{Carga de matrices de pesos
preentrenadas}\label{carga-de-matrices-de-pesos-preentrenadas}

    \begin{Verbatim}[commandchars=\\\{\}]
{\color{incolor}In [{\color{incolor}38}]:} \PY{n}{weights} \PY{o}{=} \PY{n}{loadmat}\PY{p}{(}\PY{l+s+s1}{\PYZsq{}}\PY{l+s+s1}{ex4weights.mat}\PY{l+s+s1}{\PYZsq{}}\PY{p}{)}
         \PY{n}{theta1}\PY{p}{,} \PY{n}{theta2} \PY{o}{=} \PY{n}{weights}\PY{p}{[}\PY{l+s+s1}{\PYZsq{}}\PY{l+s+s1}{Theta1}\PY{l+s+s1}{\PYZsq{}}\PY{p}{]}\PY{p}{,} \PY{n}{weights} \PY{p}{[}\PY{l+s+s1}{\PYZsq{}}\PY{l+s+s1}{Theta2}\PY{l+s+s1}{\PYZsq{}}\PY{p}{]}
\end{Verbatim}


    \subsection{Función de activación y
derivada}\label{funciuxf3n-de-activaciuxf3n-y-derivada}

En esta red neuronal utilizaremos la función sigmoide como función de
activación para las neuronas.

    \begin{Verbatim}[commandchars=\\\{\}]
{\color{incolor}In [{\color{incolor}39}]:} \PY{k}{def} \PY{n+nf}{sigmoid}\PY{p}{(}\PY{n}{z}\PY{p}{)}\PY{p}{:}
             \PY{k}{return} \PY{l+m+mi}{1}\PY{o}{/}\PY{p}{(}\PY{l+m+mi}{1} \PY{o}{+} \PY{n}{np}\PY{o}{.}\PY{n}{exp}\PY{p}{(}\PY{o}{\PYZhy{}}\PY{n}{z}\PY{p}{)}\PY{p}{)}
         \PY{k}{def} \PY{n+nf}{sigmoidDerivative}\PY{p}{(}\PY{n}{z}\PY{p}{)}\PY{p}{:}
             \PY{n}{z} \PY{o}{=} \PY{n}{sigmoid}\PY{p}{(}\PY{n}{z}\PY{p}{)}
             \PY{k}{return} \PY{n}{z}\PY{o}{*}\PY{p}{(}\PY{l+m+mi}{1}\PY{o}{\PYZhy{}}\PY{n}{z}\PY{p}{)}
         
         \PY{n+nb}{print}\PY{p}{(}\PY{l+s+s2}{\PYZdq{}}\PY{l+s+s2}{Sigmoid (0.25) = }\PY{l+s+s2}{\PYZdq{}}\PY{p}{,}\PY{n}{sigmoid}\PY{p}{(}\PY{l+m+mf}{0.25}\PY{p}{)}\PY{p}{)}
\end{Verbatim}


    \begin{Verbatim}[commandchars=\\\{\}]
Sigmoid (0.25) =  0.5621765008857981

    \end{Verbatim}

    \subsection{Forward propagation y función de
coste}\label{forward-propagation-y-funciuxf3n-de-coste}

La función de hipótesis o de forward propagation utiliza un valor de
entrada(401 entradas, con termino de sesgo ya añadido) para predecir una
salida mediante una matriz de pesos y una función de activación. Además,
durante el proceso, añadirá el termino de sesgo o \emph{bias} para el
computo final. Devolvemos todas las matrices intermedias, ya que nos
podrán ser de utilidad.

    \begin{Verbatim}[commandchars=\\\{\}]
{\color{incolor}In [{\color{incolor}40}]:} \PY{k}{def} \PY{n+nf}{forwardProp}\PY{p}{(}\PY{n}{thetas1}\PY{p}{,} \PY{n}{thetas2}\PY{p}{,} \PY{n}{X}\PY{p}{)}\PY{p}{:}
             \PY{n}{z2} \PY{o}{=} \PY{n}{thetas1}\PY{o}{.}\PY{n}{dot}\PY{p}{(}\PY{n}{X}\PY{o}{.}\PY{n}{T}\PY{p}{)}
             \PY{n}{a2} \PY{o}{=} \PY{n}{sigmoid}\PY{p}{(}\PY{n}{z2}\PY{p}{)}
             \PY{n+nb}{tuple} \PY{o}{=} \PY{p}{(}\PY{n}{np}\PY{o}{.}\PY{n}{ones}\PY{p}{(}\PY{n+nb}{len}\PY{p}{(}\PY{n}{a2}\PY{p}{[}\PY{l+m+mi}{0}\PY{p}{]}\PY{p}{)}\PY{p}{)}\PY{p}{,} \PY{n}{a2}\PY{p}{)}
             \PY{n}{a2} \PY{o}{=} \PY{n}{np}\PY{o}{.}\PY{n}{vstack}\PY{p}{(}\PY{n+nb}{tuple}\PY{p}{)}
             \PY{n}{z3} \PY{o}{=} \PY{n}{thetas2}\PY{o}{.}\PY{n}{dot}\PY{p}{(}\PY{n}{a2}\PY{p}{)}
             \PY{n}{a3} \PY{o}{=} \PY{n}{sigmoid}\PY{p}{(}\PY{n}{z3}\PY{p}{)}
             \PY{k}{return} \PY{n}{z2}\PY{p}{,} \PY{n}{a2}\PY{p}{,} \PY{n}{z3}\PY{p}{,} \PY{n}{a3}
         
         \PY{n}{X\PYZus{}aux} \PY{o}{=} \PY{n}{np}\PY{o}{.}\PY{n}{hstack}\PY{p}{(}\PY{p}{[}\PY{n}{np}\PY{o}{.}\PY{n}{ones}\PY{p}{(}\PY{p}{(}\PY{n+nb}{len}\PY{p}{(}\PY{n}{X}\PY{p}{)}\PY{p}{,} \PY{l+m+mi}{1}\PY{p}{)}\PY{p}{,} \PY{n}{dtype} \PY{o}{=} \PY{n}{np}\PY{o}{.}\PY{n}{float}\PY{p}{)}\PY{p}{,} \PY{n}{X}\PY{p}{]}\PY{p}{)}
         \PY{n+nb}{print}\PY{p}{(}\PY{l+s+s2}{\PYZdq{}}\PY{l+s+s2}{Valor predicho para el elemento 0 de X según la hipótesis: }\PY{l+s+s2}{\PYZdq{}}\PY{p}{,}\PY{p}{(}\PY{n}{forwardProp}\PY{p}{(}\PY{n}{theta1}\PY{p}{,} \PY{n}{theta2}\PY{p}{,} \PY{n}{X\PYZus{}aux}\PY{p}{)}\PY{p}{[}\PY{l+m+mi}{3}\PY{p}{]}\PY{p}{)}\PY{o}{.}\PY{n}{T}\PY{p}{[}\PY{l+m+mi}{0}\PY{p}{]}\PY{o}{.}\PY{n}{argmax}\PY{p}{(}\PY{p}{)}\PY{p}{)}
\end{Verbatim}


    \begin{Verbatim}[commandchars=\\\{\}]
Valor predicho para el elemento 0 de X según la hipótesis:  9

    \end{Verbatim}

    En cuanto a la función de coste, implementaremos la función de coste con
regularización. Como entrada a dicha función, hemos de preparar un
vector de Y distinto al recibido. Será una matriz de
\emph{(numElementos, numEtiquetas)} donde cada fila corresponde a un
caso. Cada fila tendrá todos los valores a cero menos el valor real que
representa ese caso, que estará a 1.

    \begin{Verbatim}[commandchars=\\\{\}]
{\color{incolor}In [{\color{incolor}41}]:} \PY{k}{def} \PY{n+nf}{costFun}\PY{p}{(}\PY{n}{X}\PY{p}{,} \PY{n}{y}\PY{p}{,} \PY{n}{theta1}\PY{p}{,} \PY{n}{theta2}\PY{p}{,}  \PY{n}{reg}\PY{p}{)}\PY{p}{:}
             \PY{c+c1}{\PYZsh{}Here we assert that we can operate with the parameters}
             \PY{n}{X} \PY{o}{=} \PY{n}{np}\PY{o}{.}\PY{n}{array}\PY{p}{(}\PY{n}{X}\PY{p}{)}
             \PY{n}{y} \PY{o}{=} \PY{n}{np}\PY{o}{.}\PY{n}{array}\PY{p}{(}\PY{n}{y}\PY{p}{)}
             \PY{n}{muestras} \PY{o}{=} \PY{n+nb}{len}\PY{p}{(}\PY{n}{y}\PY{p}{)}
         
             \PY{n}{theta1} \PY{o}{=} \PY{n}{np}\PY{o}{.}\PY{n}{array}\PY{p}{(}\PY{n}{theta1}\PY{p}{)}
             \PY{n}{theta2} \PY{o}{=} \PY{n}{np}\PY{o}{.}\PY{n}{array}\PY{p}{(}\PY{n}{theta2}\PY{p}{)}
         
             \PY{n}{hipo}  \PY{o}{=} \PY{n}{forwardProp}\PY{p}{(}\PY{n}{theta1}\PY{p}{,} \PY{n}{theta2}\PY{p}{,} \PY{n}{X}\PY{p}{)}\PY{p}{[}\PY{l+m+mi}{3}\PY{p}{]}
             \PY{n}{cost} \PY{o}{=} \PY{n}{np}\PY{o}{.}\PY{n}{sum}\PY{p}{(}\PY{p}{(}\PY{o}{\PYZhy{}}\PY{n}{y}\PY{o}{.}\PY{n}{T}\PY{p}{)}\PY{o}{*}\PY{p}{(}\PY{n}{np}\PY{o}{.}\PY{n}{log}\PY{p}{(}\PY{n}{hipo}\PY{p}{)}\PY{p}{)} \PY{o}{\PYZhy{}} \PY{p}{(}\PY{l+m+mi}{1}\PY{o}{\PYZhy{}}\PY{n}{y}\PY{o}{.}\PY{n}{T}\PY{p}{)}\PY{o}{*}\PY{p}{(}\PY{n}{np}\PY{o}{.}\PY{n}{log}\PY{p}{(}\PY{l+m+mi}{1}\PY{o}{\PYZhy{}} \PY{n}{hipo}\PY{p}{)}\PY{p}{)}\PY{p}{)}\PY{o}{/}\PY{n}{muestras}
         
             \PY{n}{regcost} \PY{o}{=} \PY{n}{np}\PY{o}{.}\PY{n}{sum}\PY{p}{(}\PY{n}{np}\PY{o}{.}\PY{n}{power}\PY{p}{(}\PY{n}{theta1}\PY{p}{[}\PY{p}{:}\PY{p}{,} \PY{l+m+mi}{1}\PY{p}{:}\PY{p}{]}\PY{p}{,} \PY{l+m+mi}{2}\PY{p}{)}\PY{p}{)} \PY{o}{+} \PY{n}{np}\PY{o}{.}\PY{n}{sum}\PY{p}{(}\PY{n}{np}\PY{o}{.}\PY{n}{power}\PY{p}{(}\PY{n}{theta2}\PY{p}{[}\PY{p}{:}\PY{p}{,}\PY{l+m+mi}{1}\PY{p}{:}\PY{p}{]}\PY{p}{,} \PY{l+m+mi}{2}\PY{p}{)}\PY{p}{)}
             \PY{n}{regcost} \PY{o}{=} \PY{n}{regcost} \PY{o}{*} \PY{p}{(}\PY{n}{reg}\PY{o}{/}\PY{p}{(}\PY{l+m+mi}{2}\PY{o}{*}\PY{n}{muestras}\PY{p}{)}\PY{p}{)}
         
             \PY{k}{return} \PY{n}{cost} \PY{o}{+} \PY{n}{regcost}
         
         \PY{k}{def} \PY{n+nf}{getYMatrix}\PY{p}{(}\PY{n}{Y}\PY{p}{,} \PY{n}{nEtiquetas}\PY{p}{)}\PY{p}{:}
             \PY{n}{nY} \PY{o}{=}  \PY{n}{np}\PY{o}{.}\PY{n}{zeros}\PY{p}{(}\PY{p}{(}\PY{n+nb}{len}\PY{p}{(}\PY{n}{Y}\PY{p}{)}\PY{p}{,} \PY{n}{nEtiquetas}\PY{p}{)}\PY{p}{)}
             \PY{n}{yaux} \PY{o}{=} \PY{n}{np}\PY{o}{.}\PY{n}{array}\PY{p}{(}\PY{n}{Y}\PY{p}{)} \PY{o}{\PYZhy{}}\PY{l+m+mi}{1}
             \PY{k}{for} \PY{n}{i} \PY{o+ow}{in} \PY{n+nb}{range}\PY{p}{(}\PY{n+nb}{len}\PY{p}{(}\PY{n}{nY}\PY{p}{)}\PY{p}{)}\PY{p}{:}
                 \PY{n}{z} \PY{o}{=} \PY{n}{yaux}\PY{p}{[}\PY{n}{i}\PY{p}{]}
                 \PY{k}{if}\PY{p}{(}\PY{n}{z} \PY{o}{==} \PY{l+m+mi}{10}\PY{p}{)}\PY{p}{:} \PY{n}{z} \PY{o}{=} \PY{l+m+mi}{0}
                 \PY{n}{nY}\PY{p}{[}\PY{n}{i}\PY{p}{]}\PY{p}{[}\PY{n}{z}\PY{p}{]} \PY{o}{=} \PY{l+m+mi}{1}
             \PY{k}{return} \PY{n}{nY}
\end{Verbatim}


    \begin{Verbatim}[commandchars=\\\{\}]
{\color{incolor}In [{\color{incolor}42}]:} \PY{n}{Y\PYZus{}aux} \PY{o}{=} \PY{n}{getYMatrix}\PY{p}{(}\PY{n}{Y}\PY{p}{,}\PY{l+m+mi}{10}\PY{p}{)}
\end{Verbatim}


    \begin{Verbatim}[commandchars=\\\{\}]
{\color{incolor}In [{\color{incolor}43}]:} \PY{n+nb}{print}\PY{p}{(}\PY{l+s+s2}{\PYZdq{}}\PY{l+s+s2}{El coste con thetas entrenados es: }\PY{l+s+s2}{\PYZdq{}}\PY{p}{,} \PY{n}{costFun}\PY{p}{(}\PY{n}{X\PYZus{}aux}\PY{p}{,} \PY{n}{Y\PYZus{}aux}\PY{p}{,} \PY{n}{theta1}\PY{p}{,} \PY{n}{theta2}\PY{p}{,}\PY{l+m+mi}{1}\PY{p}{)}\PY{p}{)}
\end{Verbatim}


    \begin{Verbatim}[commandchars=\\\{\}]
El coste con thetas entrenados es:  0.3837698590909236

    \end{Verbatim}

    \subsection{Backpropagation}\label{backpropagation}

Función de backpropagation para repartir el error entre las neuronas de
la red neuronal. Comienza desde la ultima capa y desde esa desciende
hasta la penúltima, ya que no se puede repartir error para la capa de
entrada.

    \begin{Verbatim}[commandchars=\\\{\}]
{\color{incolor}In [{\color{incolor}44}]:} \PY{k}{def} \PY{n+nf}{backprop}\PY{p}{(}\PY{n}{params\PYZus{}rn}\PY{p}{,} \PY{n}{num\PYZus{}entradas}\PY{p}{,} \PY{n}{num\PYZus{}ocultas}\PY{p}{,} \PY{n}{num\PYZus{}etiquetas}\PY{p}{,} \PY{n}{X}\PY{p}{,} \PY{n}{Y}\PY{p}{,} \PY{n}{reg}\PY{p}{)}\PY{p}{:}
             \PY{n}{th1} \PY{o}{=} \PY{n}{np}\PY{o}{.}\PY{n}{reshape}\PY{p}{(}\PY{n}{params\PYZus{}rn}\PY{p}{[}\PY{p}{:}\PY{n}{num\PYZus{}ocultas} \PY{o}{*}\PY{p}{(}\PY{n}{num\PYZus{}entradas} \PY{o}{+} \PY{l+m+mi}{1}\PY{p}{)}\PY{p}{]}\PY{p}{,}\PY{p}{(}\PY{n}{num\PYZus{}ocultas}\PY{p}{,} \PY{p}{(}\PY{n}{num\PYZus{}entradas}\PY{o}{+}\PY{l+m+mi}{1}\PY{p}{)}\PY{p}{)}\PY{p}{)}
             \PY{c+c1}{\PYZsh{} theta2 es un array de (num\PYZus{}etiquetas, num\PYZus{}ocultas)}
             \PY{n}{th2} \PY{o}{=} \PY{n}{np}\PY{o}{.}\PY{n}{reshape}\PY{p}{(}\PY{n}{params\PYZus{}rn}\PY{p}{[}\PY{n}{num\PYZus{}ocultas}\PY{o}{*}\PY{p}{(}\PY{n}{num\PYZus{}entradas} \PY{o}{+} \PY{l+m+mi}{1}\PY{p}{)}\PY{p}{:} \PY{p}{]}\PY{p}{,} \PY{p}{(}\PY{n}{num\PYZus{}etiquetas}\PY{p}{,}\PY{p}{(}\PY{n}{num\PYZus{}ocultas}\PY{o}{+}\PY{l+m+mi}{1}\PY{p}{)}\PY{p}{)}\PY{p}{)}
             
             \PY{n}{X\PYZus{}unos} \PY{o}{=} \PY{n}{np}\PY{o}{.}\PY{n}{hstack}\PY{p}{(}\PY{p}{[}\PY{n}{np}\PY{o}{.}\PY{n}{ones}\PY{p}{(}\PY{p}{(}\PY{n+nb}{len}\PY{p}{(}\PY{n}{X}\PY{p}{)}\PY{p}{,} \PY{l+m+mi}{1}\PY{p}{)}\PY{p}{,} \PY{n}{dtype} \PY{o}{=} \PY{n}{np}\PY{o}{.}\PY{n}{float}\PY{p}{)}\PY{p}{,} \PY{n}{X}\PY{p}{]}\PY{p}{)}
             \PY{n}{nMuestras} \PY{o}{=} \PY{n+nb}{len}\PY{p}{(}\PY{n}{X}\PY{p}{)}
             
             \PY{n}{y} \PY{o}{=} \PY{n}{np}\PY{o}{.}\PY{n}{zeros}\PY{p}{(}\PY{p}{(}\PY{n}{nMuestras}\PY{p}{,} \PY{n}{num\PYZus{}etiquetas}\PY{p}{)}\PY{p}{)}
             \PY{n}{y} \PY{o}{=} \PY{n}{getYMatrix}\PY{p}{(}\PY{n}{Y}\PY{p}{,} \PY{n}{num\PYZus{}etiquetas}\PY{p}{)}
             
             \PY{n}{coste} \PY{o}{=} \PY{n}{costFun}\PY{p}{(}\PY{n}{X\PYZus{}unos}\PY{p}{,} \PY{n}{y}\PY{p}{,} \PY{n}{th1}\PY{p}{,} \PY{n}{th2}\PY{p}{,} \PY{n}{reg}\PY{p}{)}
             
             
             
             \PY{c+c1}{\PYZsh{}Backpropagation}
             
             \PY{c+c1}{\PYZsh{} Forward propagation para obtener una hipótesis y los valores intermedios}
             \PY{c+c1}{\PYZsh{} de la red neuronal}
             \PY{n}{z2}\PY{p}{,} \PY{n}{a2}\PY{p}{,} \PY{n}{z3}\PY{p}{,} \PY{n}{a3} \PY{o}{=} \PY{n}{forwardProp}\PY{p}{(}\PY{n}{th1}\PY{p}{,} \PY{n}{th2}\PY{p}{,} \PY{n}{X\PYZus{}unos}\PY{p}{)}
             
             \PY{n}{gradW1} \PY{o}{=} \PY{n}{np}\PY{o}{.}\PY{n}{zeros}\PY{p}{(}\PY{n}{th1}\PY{o}{.}\PY{n}{shape}\PY{p}{)}
             \PY{n}{gradW2} \PY{o}{=} \PY{n}{np}\PY{o}{.}\PY{n}{zeros}\PY{p}{(}\PY{n}{th2}\PY{o}{.}\PY{n}{shape}\PY{p}{)}
             
             \PY{c+c1}{\PYZsh{} Coste por capas}
             \PY{n}{delta3} \PY{o}{=} \PY{n}{np}\PY{o}{.}\PY{n}{array}\PY{p}{(}\PY{n}{a3} \PY{o}{\PYZhy{}} \PY{n}{y}\PY{o}{.}\PY{n}{T}\PY{p}{)}
             \PY{n}{delta2} \PY{o}{=} \PY{n}{th2}\PY{o}{.}\PY{n}{T}\PY{p}{[}\PY{l+m+mi}{1}\PY{p}{:}\PY{p}{,}\PY{p}{:}\PY{p}{]}\PY{o}{.}\PY{n}{dot}\PY{p}{(}\PY{n}{delta3}\PY{p}{)}\PY{o}{*}\PY{n}{sigmoidDerivative}\PY{p}{(}\PY{n}{z2}\PY{p}{)}
         
             
             \PY{c+c1}{\PYZsh{} Acumulación de gradiente}
             \PY{n}{gradW1} \PY{o}{=} \PY{n}{gradW1} \PY{o}{+} \PY{p}{(}\PY{n}{delta2}\PY{o}{.}\PY{n}{dot}\PY{p}{(}\PY{n}{X\PYZus{}unos}\PY{p}{)}\PY{p}{)}
             \PY{n}{gradW2} \PY{o}{=} \PY{n}{gradW2} \PY{o}{+} \PY{p}{(}\PY{n}{delta3}\PY{o}{.}\PY{n}{dot}\PY{p}{(}\PY{n}{a2}\PY{o}{.}\PY{n}{T}\PY{p}{)}\PY{p}{)}
         
             
             \PY{n}{G1} \PY{o}{=} \PY{n}{gradW1}\PY{o}{/}\PY{n+nb}{float}\PY{p}{(}\PY{n}{nMuestras}\PY{p}{)}
             \PY{n}{G2} \PY{o}{=} \PY{n}{gradW2}\PY{o}{/}\PY{n+nb}{float}\PY{p}{(}\PY{n}{nMuestras}\PY{p}{)}
         
             \PY{c+c1}{\PYZsh{}suma definitiva}
             \PY{n}{G1}\PY{p}{[}\PY{p}{:}\PY{p}{,} \PY{l+m+mi}{1}\PY{p}{:} \PY{p}{]} \PY{o}{=} \PY{n}{G1}\PY{p}{[}\PY{p}{:}\PY{p}{,} \PY{l+m+mi}{1}\PY{p}{:}\PY{p}{]} \PY{o}{+} \PY{p}{(}\PY{n+nb}{float}\PY{p}{(}\PY{n}{reg}\PY{p}{)}\PY{o}{/}\PY{n+nb}{float}\PY{p}{(}\PY{n}{nMuestras}\PY{p}{)}\PY{p}{)}\PY{o}{*}\PY{n}{th1}\PY{p}{[}\PY{p}{:}\PY{p}{,} \PY{l+m+mi}{1}\PY{p}{:}\PY{p}{]}
             \PY{n}{G2}\PY{p}{[}\PY{p}{:}\PY{p}{,} \PY{l+m+mi}{1}\PY{p}{:} \PY{p}{]} \PY{o}{=} \PY{n}{G2}\PY{p}{[}\PY{p}{:}\PY{p}{,} \PY{l+m+mi}{1}\PY{p}{:}\PY{p}{]} \PY{o}{+} \PY{p}{(}\PY{n+nb}{float}\PY{p}{(}\PY{n}{reg}\PY{p}{)}\PY{o}{/}\PY{n+nb}{float}\PY{p}{(}\PY{n}{nMuestras}\PY{p}{)}\PY{p}{)}\PY{o}{*}\PY{n}{th2}\PY{p}{[}\PY{p}{:}\PY{p}{,} \PY{l+m+mi}{1}\PY{p}{:}\PY{p}{]}
             
             
             \PY{n}{gradients} \PY{o}{=} \PY{n}{np}\PY{o}{.}\PY{n}{concatenate}\PY{p}{(}\PY{p}{(}\PY{n}{G1}\PY{p}{,} \PY{n}{G2}\PY{p}{)}\PY{p}{,} \PY{n}{axis} \PY{o}{=} \PY{k+kc}{None}\PY{p}{)}
             
             \PY{k}{return} \PY{n}{coste}\PY{p}{,} \PY{n}{gradients}
             
\end{Verbatim}


    \begin{Verbatim}[commandchars=\\\{\}]
{\color{incolor}In [{\color{incolor}45}]:} \PY{n}{params} \PY{o}{=}  \PY{n}{np}\PY{o}{.}\PY{n}{concatenate}\PY{p}{(}\PY{p}{(}\PY{n}{theta1}\PY{p}{,} \PY{n}{theta2}\PY{p}{)}\PY{p}{,} \PY{n}{axis} \PY{o}{=} \PY{k+kc}{None}\PY{p}{)}
         
         \PY{c+c1}{\PYZsh{}print(backprop(params, 400, 25, 10, X, Y, 1))}
         \PY{n+nb}{print}\PY{p}{(}\PY{l+s+s2}{\PYZdq{}}\PY{l+s+s2}{Diferencias al comprobar gradientes:}\PY{l+s+se}{\PYZbs{}n}\PY{l+s+s2}{\PYZdq{}}\PY{p}{,} \PY{n}{check}\PY{o}{.}\PY{n}{checkNNGradients}\PY{p}{(}\PY{n}{backprop}\PY{p}{,} \PY{l+m+mi}{1}\PY{p}{)}\PY{p}{)}
\end{Verbatim}


    \begin{Verbatim}[commandchars=\\\{\}]
Diferencias al comprobar gradientes:
 [ 4.33315606e-11 -5.85087534e-13  5.24080779e-13  6.94293928e-12
 -3.86019966e-11  8.55844562e-12 -7.97453770e-12 -3.25843796e-11
 -5.90238414e-11  3.02491365e-11 -2.21222485e-11 -9.52720680e-11
 -4.15551621e-11  9.12638021e-13 -2.03395634e-12 -1.80884266e-11
  1.25427238e-11 -4.09060286e-12  6.03773281e-12  2.41384690e-11
  5.28279642e-11  1.03140274e-11  6.16659501e-12  8.66137717e-12
  9.34252675e-12  1.71084258e-11  6.32419117e-11  1.16325005e-11
  1.02217609e-11  1.49442403e-11  1.12843068e-11  1.66834324e-11
  7.14504289e-11  1.10622622e-11  8.72196759e-12  2.17894591e-11
  8.18731194e-12  1.53632801e-11]

    \end{Verbatim}

    \#\#\# Inicialización aleatoria de thetas

    \begin{Verbatim}[commandchars=\\\{\}]
{\color{incolor}In [{\color{incolor}46}]:} \PY{k}{def} \PY{n+nf}{weightInitialize}\PY{p}{(}\PY{n}{L\PYZus{}in}\PY{p}{,} \PY{n}{L\PYZus{}out}\PY{p}{)}\PY{p}{:}
             \PY{n}{cini} \PY{o}{=} \PY{l+m+mf}{0.12}
             \PY{n}{aux} \PY{o}{=} \PY{n}{np}\PY{o}{.}\PY{n}{random}\PY{o}{.}\PY{n}{uniform}\PY{p}{(}\PY{o}{\PYZhy{}}\PY{n}{cini}\PY{p}{,} \PY{n}{cini}\PY{p}{,} \PY{n}{size} \PY{o}{=}\PY{p}{(}\PY{n}{L\PYZus{}in}\PY{p}{,} \PY{n}{L\PYZus{}out}\PY{p}{)}\PY{p}{)}
             \PY{n}{aux} \PY{o}{=} \PY{n}{np}\PY{o}{.}\PY{n}{insert}\PY{p}{(}\PY{n}{aux}\PY{p}{,}\PY{l+m+mi}{0}\PY{p}{,}\PY{l+m+mi}{1}\PY{p}{,}\PY{n}{axis} \PY{o}{=} \PY{l+m+mi}{0}\PY{p}{)}
             \PY{k}{return} \PY{n}{aux}
\end{Verbatim}


    \subsection{Prueba para la red
Neuronal}\label{prueba-para-la-red-neuronal}

Con esta función probaremos la red con matrices de pesos inicializadas
aleatoriamente y comprobaremos su precisión después de ser optimizada
con la función optimize.

    \begin{Verbatim}[commandchars=\\\{\}]
{\color{incolor}In [{\color{incolor}63}]:} \PY{k}{def} \PY{n+nf}{NNTest} \PY{p}{(}\PY{n}{num\PYZus{}entradas}\PY{p}{,} \PY{n}{num\PYZus{}ocultas}\PY{p}{,} \PY{n}{num\PYZus{}etiquetas}\PY{p}{,} \PY{n}{reg}\PY{p}{,} \PY{n}{X}\PY{p}{,} \PY{n}{Y}\PY{p}{,} \PY{n}{laps}\PY{p}{)}\PY{p}{:}
             \PY{n}{t1} \PY{o}{=} \PY{n}{weightInitialize}\PY{p}{(}\PY{n}{num\PYZus{}entradas}\PY{p}{,} \PY{n}{num\PYZus{}ocultas}\PY{p}{)}
             \PY{n}{t2} \PY{o}{=} \PY{n}{weightInitialize}\PY{p}{(}\PY{n}{num\PYZus{}ocultas}\PY{p}{,} \PY{n}{num\PYZus{}etiquetas}\PY{p}{)}
         
             \PY{n}{params} \PY{o}{=} \PY{n}{np}\PY{o}{.}\PY{n}{hstack}\PY{p}{(}\PY{p}{(}\PY{n}{np}\PY{o}{.}\PY{n}{ravel}\PY{p}{(}\PY{n}{t1}\PY{p}{)}\PY{p}{,} \PY{n}{np}\PY{o}{.}\PY{n}{ravel}\PY{p}{(}\PY{n}{t2}\PY{p}{)}\PY{p}{)}\PY{p}{)}
             \PY{n}{out} \PY{o}{=} \PY{n}{opt}\PY{o}{.}\PY{n}{minimize}\PY{p}{(}\PY{n}{fun} \PY{o}{=} \PY{n}{backprop}\PY{p}{,} \PY{n}{x0} \PY{o}{=} \PY{n}{params}\PY{p}{,} \PY{n}{args} \PY{o}{=} \PY{p}{(}\PY{n}{num\PYZus{}entradas}\PY{p}{,} \PY{n}{num\PYZus{}ocultas}\PY{p}{,} \PY{n}{num\PYZus{}etiquetas}\PY{p}{,} \PY{n}{X}\PY{p}{,} \PY{n}{Y}\PY{p}{,} \PY{n}{reg}\PY{p}{)}\PY{p}{,} \PY{n}{method}\PY{o}{=}\PY{l+s+s1}{\PYZsq{}}\PY{l+s+s1}{TNC}\PY{l+s+s1}{\PYZsq{}}\PY{p}{,} \PY{n}{jac} \PY{o}{=} \PY{k+kc}{True}\PY{p}{,} \PY{n}{options} \PY{o}{=} \PY{p}{\PYZob{}}\PY{l+s+s1}{\PYZsq{}}\PY{l+s+s1}{maxiter}\PY{l+s+s1}{\PYZsq{}}\PY{p}{:} \PY{n}{laps}\PY{p}{\PYZcb{}}\PY{p}{)}
         
             \PY{n}{Thetas1} \PY{o}{=} \PY{n}{out}\PY{o}{.}\PY{n}{x}\PY{p}{[}\PY{p}{:}\PY{p}{(}\PY{n}{num\PYZus{}ocultas}\PY{o}{*}\PY{p}{(}\PY{n}{num\PYZus{}entradas}\PY{o}{+}\PY{l+m+mi}{1}\PY{p}{)}\PY{p}{)}\PY{p}{]}\PY{o}{.}\PY{n}{reshape}\PY{p}{(}\PY{n}{num\PYZus{}ocultas}\PY{p}{,}\PY{p}{(}\PY{n}{num\PYZus{}entradas}\PY{o}{+}\PY{l+m+mi}{1}\PY{p}{)}\PY{p}{)}
             \PY{n}{Thetas2} \PY{o}{=} \PY{n}{out}\PY{o}{.}\PY{n}{x}\PY{p}{[}\PY{p}{(}\PY{n}{num\PYZus{}ocultas}\PY{o}{*}\PY{p}{(}\PY{n}{num\PYZus{}entradas}\PY{o}{+}\PY{l+m+mi}{1}\PY{p}{)}\PY{p}{)}\PY{p}{:}\PY{p}{]}\PY{o}{.}\PY{n}{reshape}\PY{p}{(}\PY{n}{num\PYZus{}etiquetas}\PY{p}{,}\PY{p}{(}\PY{n}{num\PYZus{}ocultas}\PY{o}{+}\PY{l+m+mi}{1}\PY{p}{)}\PY{p}{)}
         
             \PY{n+nb}{input} \PY{o}{=} \PY{n}{np}\PY{o}{.}\PY{n}{hstack}\PY{p}{(}\PY{p}{[}\PY{n}{np}\PY{o}{.}\PY{n}{ones}\PY{p}{(}\PY{p}{(}\PY{n+nb}{len}\PY{p}{(}\PY{n}{X}\PY{p}{)}\PY{p}{,} \PY{l+m+mi}{1}\PY{p}{)}\PY{p}{,} \PY{n}{dtype} \PY{o}{=} \PY{n}{np}\PY{o}{.}\PY{n}{float}\PY{p}{)}\PY{p}{,} \PY{n}{X}\PY{p}{]}\PY{p}{)}
             \PY{n}{hipo} \PY{o}{=} \PY{n}{forwardProp}\PY{p}{(}\PY{n}{Thetas1}\PY{p}{,} \PY{n}{Thetas2}\PY{p}{,} \PY{n+nb}{input}\PY{p}{)}\PY{p}{[}\PY{l+m+mi}{3}\PY{p}{]}
         
         
             \PY{n}{Ghipo} \PY{o}{=} \PY{p}{(}\PY{n}{hipo}\PY{o}{.}\PY{n}{argmax}\PY{p}{(}\PY{n}{axis} \PY{o}{=} \PY{l+m+mi}{0}\PY{p}{)}\PY{p}{)}\PY{o}{+}\PY{l+m+mi}{1}
             \PY{n}{prec} \PY{o}{=} \PY{p}{(}\PY{n}{Ghipo} \PY{o}{==} \PY{n}{Y}\PY{p}{)}\PY{o}{*}\PY{l+m+mi}{1}
             
             \PY{n}{precision} \PY{o}{=} \PY{n+nb}{sum}\PY{p}{(}\PY{n}{prec}\PY{p}{)} \PY{o}{/} \PY{n+nb}{len}\PY{p}{(}\PY{n}{X}\PY{p}{)}
         
             \PY{n+nb}{print}\PY{p}{(}\PY{l+s+s2}{\PYZdq{}}\PY{l+s+s2}{Program precision: }\PY{l+s+s2}{\PYZdq{}}\PY{p}{,} \PY{n}{precision} \PY{o}{*}\PY{l+m+mi}{100}\PY{p}{,} \PY{l+s+s2}{\PYZdq{}}\PY{l+s+s2}{\PYZpc{}}\PY{l+s+s2}{\PYZdq{}}\PY{p}{)}
\end{Verbatim}


    \begin{Verbatim}[commandchars=\\\{\}]
{\color{incolor}In [{\color{incolor}67}]:} \PY{n}{NNTest}\PY{p}{(}\PY{l+m+mi}{400}\PY{p}{,} \PY{l+m+mi}{25}\PY{p}{,} \PY{l+m+mi}{10}\PY{p}{,} \PY{l+m+mi}{1}\PY{p}{,} \PY{n}{X}\PY{p}{,} \PY{n}{Y}\PY{p}{,} \PY{l+m+mi}{70}\PY{p}{)}
\end{Verbatim}


    \begin{Verbatim}[commandchars=\\\{\}]
Program precision:  91.8 \%

    \end{Verbatim}


    % Add a bibliography block to the postdoc
    
    
    
    \end{document}
